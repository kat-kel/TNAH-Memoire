% PREAMBULE
% !BIB TS-program = biber
% !TEX TS-program = xelatexmk
% ITEX TS-program = latex
% !TEX spellcheck = French

%%%%%%%%%%%%%%%%%%%%%%%%
%			TYPE DE DOCUMENT
\documentclass[a4paper,12pt,twoside]{book}

%%%%%%%%%%%%%%%%%%%%%%%%
%			MULTI-FILE
\usepackage{blindtext}
\usepackage{subfiles}

%%%%%%%%%%%%%%%%%%%%%%%%
%			ENCODAGE
\usepackage{fontspec}
\usepackage{lettrine} % lettrines (pas obligatoire)
\usepackage{caption}
\usepackage{subcaption}

%%%%%%%%%%%%%%%%%%%%%%%%
%			REFERENCES
% le package hyperref avec des options, si en local
\usepackage{hyperref}
\usepackage[backend=biber, sorting=nyt, style=verbose-ibid]{biblatex}
\addbibresource{../bib.bib}

%%%%%%%%%%%%%%%%%%%%%%%%
%			LANGUE
% !TEX spellcheck = French
\usepackage[french]{babel}
\usepackage{csquotes}

%%%%%%%%%%%%%%%%%%%%%%%%
%			ENC
\usepackage[margin=2.5cm]{geometry} %marges
\usepackage{setspace} % espacement qui permet ensuite de définir un interligne
\onehalfspacing % interligne de 1.5
\setlength\parindent{1cm} % indentation des paragraphes à 1 cm

%%%%%%%%%%%%%%%%%%%%%%%%
%			GLOSSAIRE
\usepackage[acronym]{glossaries}
\makeglossaries
\newglossaryentry{htr}
{
    name=Handwritten Text Recognition,
    description={La reconnaissance du texte écrit sur une image numérique}
}
\newacronym{HTR}{HTR}{Handwritten Text Recognition}

\newglossaryentry{ocr}
{
    name=Optical Character Recognition,
    description={La reconnaissance des polices du texte sur une image numérique}
}
\newacronym{OCR}{OCR}{Optical Character Recognition}

\newglossaryentry{Inria}
{
    name=Inria,
    description={Institut national de recherche en sciences et technologies du numérique}
}
\newacronym{INRIA}{Inria}{Institut national de recherche en sciences et technologies du numérique}

\newacronym{almanach}{ALMAnaCH}{Automatic Language Modelling and Analysis \& Computational Humanities}

\newglossaryentry{enc}
{
    name=École nationale des chartes,
    description={Grande école bla bla bla}
}
\newacronym{ENC}{ENC}{École nationale des chartes}

\newglossaryentry{HTR-United}
{
    name=HTR-United,
    description={HTR-United is a catalog and an ecosystem for sharing and finding ground truth for optical character or handwritten text recognition (OCR/HTR)}
}

\newglossaryentry{CLab}
{
	name=CREMMALab,
	description={Consortium pour la reconnaissance
d’'écriture manuscrite des matériaux anciens}
}
\newacronym{CREMMA}{CREMMA}{Consortium Reconnaissance
d’Écriture Manuscrite des Matériaux Anciens}

\newglossaryentry{tei}
{
	name={Text Encoding Initiative},
	description={Normes internationales de l'encodage des documents texts}
}
\newacronym{TEI}{TEI}{Text Encoding Initiative}

\newglossaryentry{iiif}
{
	name={International Image Interoperability Framework},
	description={Normes internationales de l'exploitation des images numériques et de leurs métadonnées par API}
}
\newacronym{IIIF}{IIIF}{International Image Interoperability Framework}

\newacronym{ALTO}{ALTO}{Analyzed Layout and Text Object}

\newacronym{XML}{XML}{eXtensible Markup Language}

\newacronym{BNF}{BnF}{Bibliothèque nationale de France}

\newacronym{RDF}{RDF}{Resource Description Framework}

\newacronym{TAL}{TAL}{Traitement automatique des langues}

\newacronym{ARK}{ARK}{Archival Resource Key}

\newacronym{DTS}{DTS}{Distributed Text Services}

\newglossaryentry{iiifapi}
{
	name={IIIF Image API},
	description={Un service de web qui renvoie une image suite à une requête standardisée HTTP(S). L'URI peut préciser la région, la taille, la rotation, la qualité, les caractéristiques, et le format de l'image demandée.}
}
\newacronym{API}{API}{Application Programming Interface}

\newglossaryentry{odd}
{
	name={One Document Does it all},
	description={Un fichier XML TEI qui précise les règles d'un schème TEI personnalisé.}
}
\newacronym{ODD}{ODD}{One Document Does it all}

\newacronym{JSON}{JSON}{JavaScript Object Notation}

\newacronym{HTML}{HTML}{HyperText Markup Language}

\newacronym{METS}{METS}{Metadata Encoding and Transmission Standard}

\newacronym{YAML}{YAML}{Yet Another Markup Language}

\newacronym{SRU}{SRU}{Search/Retrieve via URL}

\newglossaryentry{unimarc}
{
	name={UNIMARC},
	description={Une référence pour l’échange de données en format XML}
}

\newacronym{SUDOC}{SUDOC}{Système Universitaire de Documentation}

%%%%%%%%%%%%%%%%%%%%%%%%
%			DIAGRAM
\usepackage{tikz}
\usetikzlibrary{positioning}
\usetikzlibrary{calc, matrix, shapes.geometric, arrows}
\usepackage{pgfplots}
\usepackage{array}
\usepackage{tabularx}

%si index, package pour index + makeindex

%%%%%%%%%%%%%%%%%%%%%%%%
%			EXEMPLES DE CODE
\usepackage{listings}
	\usepackage{color}
	\definecolor{codegray}{rgb}{0.5,0.5,0.5}
	\definecolor{codepurple}{rgb}{0.58,0,0.82}
	\definecolor{cyan}{rgb}{0.0,0.6,0.6}
	\definecolor{codegreen}{rgb}{0,0.6,0}
	\definecolor{backcolour}{rgb}{0.95,0.95,0.92}

	\lstdefinelanguage{XML}{
  	backgroundcolor=\color{backcolour},  
  	basicstyle=\ttfamily\footnotesize,
  	morestring=[s]{"}{"},
  	morecomment=[s]{?}{?},
  	morecomment=[s]{!--}{--},
  	commentstyle=\color{codegreen},
  	moredelim=[s][\color{black}]{>}{<},
  	moredelim=[s][\color{red}]{\ }{=},
  	stringstyle=\color{blue},
  	identifierstyle=\color{codegray},
  	numberstyle=\tiny\color{codegray},
  	breakatwhitespace=false,         
    	breaklines=true,                 
    	captionpos=b,                    
    	keepspaces=true,                 
    	numbers=left,                    
    	numbersep=5pt,                  
    	showspaces=false,                
    	showstringspaces=false,
    	showtabs=false,                  
    	tabsize=2
	}

	\lstdefinestyle{json}{
  	backgroundcolor=\color{backcolour}, 
  	basicstyle=\ttfamily\footnotesize,
  	commentstyle=\color{codegreen},
  	numberstyle=\tiny\color{codegray},
  	basicstyle=\ttfamily\footnotesize,
  	breakatwhitespace=false,         
  	breaklines=true,                 
  	captionpos=b,                    
  	keepspaces=true,                 
  	numbers=left,                    
  	numbersep=5pt,                  
  	showspaces=false,                
  	showstringspaces=false,
  	showtabs=false,                  
  	tabsize=2
	}

%%%%%%%%%%%%%%%%%%%%%%%%
%			IMAGES
\usepackage{graphicx}

%%%%%%%%%%%%%%%%%%%%%%%%
%			AUTEUR.RICE & TITRE
\author{Kelly Christensen}
\title{D’ALTO à TEI, modélisation de transcriptions automatiques pour une pré-éditorialisant des textes}


%%%%%%%%%%%%%%%%%%%%%%%%
%%%%%%%%%%%%%%%%%%%%%%%%
%			DOCUMENT
%%%%%%%%%%%%%%%%%%%%%%%%
%%%%%%%%%%%%%%%%%%%%%%%%
\begin{document}
	\begin{titlepage}
		\begin{center}
			
			\bigskip
			
			\begin{large}				
				ÉCOLE NATIONALE DES CHARTES\\
				UNIVERSITÉ PARIS, SCIENCES \& LETTRES
			\end{large}
			\begin{center}\rule{2cm}{0.02cm}\end{center}
			
			\bigskip
			\bigskip
			\bigskip
			\begin{Large}
				\textbf{Kelly Christensen}\\
			\end{Large}
		%selon le cas
			\begin{normalsize} 
				\textit{diplômée de doctorat musicologie}
			\end{normalsize}
			
			\bigskip
			\bigskip
			\bigskip
			
			\begin{Huge}
				\textbf{D’ALTO à TEI}\\
			\end{Huge}
			\bigskip
			\bigskip
			\begin{LARGE}
				\textbf{Modélisation de transcriptions automatiques pour une pré-éditorialisant des textes}\\
			\end{LARGE}
			
			\bigskip
			\bigskip
			\bigskip
			\begin{large}
			\end{large}
			\vfill
			
			\begin{large}
				Mémoire 
				pour le diplôme de master \\
				\og{} Technologies numériques appliquées à l'histoire \fg{} \\
				\bigskip
				2022
			\end{large}
			
		\end{center}
	\end{titlepage}
	
	\thispagestyle{empty}	
	\cleardoublepage
	
%%%%%%%%%%%%%%%%%%%%%%%%
%			FRONT MATTER
	\frontmatter
	\cleardoublepage
	
	\chapter{Résumé}
	
	\subfile{sections/abstract/abstract.tex}
	
	\chapter{Remerciements}
	\lettrine{M}es remerciements vont tout d'abord à\dots
	
	%bibliographie
	\printbibliography{ref}
	
	%glossaire / acronymes
	\printglossaries
	
	\chapter{Introduction}
	\subfile{sections/intro/introduction.tex}
	
	\thispagestyle{empty}
	\cleardoublepage
	
%%%%%%%%%%%%%%%%%%%%%%%%
%			MAIN MATTER
	\mainmatter
	
%%%%%%%%%%%%%%%%%%%%%%%%
%			PARTIE I.
	\part{Présentation du projet}
	
	\chapter{Le rêve du projet \textit{Gallic(orpor)a}}
	\subfile{sections/chap3/chap3.tex}
	
	\chapter{Au commencement, il y avait les \textit{guidelines SegmOnto}}
	\label{chap:segmonto}
	\subfile{sections/chap2/chap2.tex}
	
	\chapter{Qu'est-ce que l'HTR ?}
	\subfile{sections/chap1/chap1.tex}
	\label{chap:htr}
	
%%%%%%%%%%%%%%%%%%%%%%%%
%			PARTIE II.
	\part{Exposition de la préparation et du travail d'analyse}

	\chapter{Un pipeline visant à tout rassembler}
	\subfile{sections/chap4/chap4.tex}
	
	\chapter{L'analyse des structures des données XML}
	\subfile{sections/chap5/chap5.tex}
	\label{chap:xml}
	
	\chapter{À la recherche des métadonnées}
	\subfile{sections/chap6/chap6.tex}
	
%%%%%%%%%%%%%%%%%%%%%%%%
%			PARTIE III.
	\part{Mise en opérationnelle du projet}
	
	\chapter{La génération du \texttt{<teiHeader>}}
	\subfile{sections/chap7/chap7.tex}
	
	\chapter{La modélisation de la \texttt{<sourceDoc>}}
	\subfile{sections/chap8/chap8.tex}
	
	\chapter{Les données textuelles produites}
	\subfile{sections/chap9/chap9.tex}
	
	\chapter*{Conclusion}
	\subfile{sections/conclusion/conclusion.tex}
	
	\addcontentsline{toc}{chapter}{Conclusion}
	
%%%%%%%%%%%%%%%%%%%%%%%%
%			ANNEXES
	\appendix
	\chapter{Données}
	\subfile{sections/data/data.tex}
	
	\backmatter

%%%%%%%%%%%%%%%%%%%%%%%%
%			INDEX
%	\printindex

%%%%%%%%%%%%%%%%%%%%%%%%
%			FIGURES
	\listoffigures

%%%%%%%%%%%%%%%%%%%%%%%%
%			TABLES
	\listoftables

%%%%%%%%%%%%%%%%%%%%%%%%
%			TABLE DE MATIERES
	\tableofcontents
	
\end{document}