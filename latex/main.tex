% document vide aux normes de l'école pour le mémoire

% PREAMBULE
% !TEX TS-program = xelatex
% !TEX spellcheck = French

%package obligatoire : type de document
\documentclass[a4paper,12pt,twoside]{book}

% multi-file
\usepackage{blindtext}
\usepackage{subfiles}

% encodage
\usepackage{fontspec}

% le package hyperref avec des options, si en local
\usepackage[pdfusetitle, pdfsubject ={Mémoire TNAH}, pdfkeywords={les mots-clés}]{hyperref}
\usepackage{hyperref}

%il faut mettre au moins une langue
\usepackage[french]{babel}
\usepackage{csquotes}

% configurer le document selon les normes de l'école
\usepackage[margin=2.5cm]{geometry} %marges
\usepackage{setspace} % espacement qui permet ensuite de définir un interligne
\onehalfspacing % interligne de 1.5
\setlength\parindent{1cm} % indentation des paragraphes à 1 cm

\usepackage{lettrine} % lettrines (pas obligatoire)

% gloss
\usepackage{glossaries}
\makeglossaries

\newglossaryentry{schema}
{
	name=ALTO,
	description={Un schéma XML qui précise la mise en page ainsi que le contenu d'une ressource textuelle, tel qu'un bouquin.}
	name=TEI,
	description={Un schéma XML qui résume des règles et des normes d'encodage qui ont pour but de rendre une source textuelle lisible aux machines}
}
\newacronym{htr}{HTR}{Handwritten Text Recognition}
\newacronym{ocr}{OCR}{Optical Character Recognition}
\newacronym{alto}{ALTO}{Analyzed Layout and Text Object}
\newacronym{tei}{TEI}{Text Encoding Initiaite}
\newacronym{xml}{XML}{eXtensible Markup Language}
\newacronym{bnf}{BnF}{Bibliothèque nationale de France}
\newacronym{almanach}{ALMAnaCH}{Automatic Language Modelling and Analysis \& Computational Humanities}
\newacronym{rdf}{RDF}{Resource Description Framework}


% bibliographie
\usepackage[backend=biber, sorting=nyt, style=enc,maxbibnames=10]{biblatex}
\addbibresource{../bib.bib}

%si index, package pour index + makeindex

% flowchart
\usepackage{tikz}
\usetikzlibrary{shapes.geometric, arrows}
\tikzstyle{startstop} = [rectangle, rounded corners, minimum width=3cm, minimum height=1cm,text centered, draw=black, fill=red!30]
\tikzstyle{arrow} = [thick,->,>=stealth]

% Exemples de code
\usepackage{listings}
\usepackage{color}
\definecolor{codegray}{rgb}{0.5,0.5,0.5}
\definecolor{codepurple}{rgb}{0.58,0,0.82}
\definecolor{cyan}{rgb}{0.0,0.6,0.6}
\definecolor{codegreen}{rgb}{0,0.6,0}
\definecolor{backcolour}{rgb}{0.95,0.95,0.92}

\lstdefinelanguage{XML}{
  backgroundcolor=\color{backcolour},  
  basicstyle=\ttfamily\footnotesize,
  morestring=[s]{"}{"},
  morecomment=[s]{?}{?},
  morecomment=[s]{!--}{--},
  commentstyle=\color{codegreen},
  moredelim=[s][\color{black}]{>}{<},
  moredelim=[s][\color{red}]{\ }{=},
  stringstyle=\color{blue},
  identifierstyle=\color{codegray},
  numberstyle=\tiny\color{codegray},
  breakatwhitespace=false,         
    breaklines=true,                 
    captionpos=b,                    
    keepspaces=true,                 
    numbers=left,                    
    numbersep=5pt,                  
    showspaces=false,                
    showstringspaces=false,
    showtabs=false,                  
    tabsize=2
}

% Code listing style "json"
\lstdefinestyle{json}{
  backgroundcolor=\color{backcolour}, 
  basicstyle=\ttfamily\footnotesize,
  commentstyle=\color{codegreen},
  numberstyle=\tiny\color{codegray},
  basicstyle=\ttfamily\footnotesize,
  breakatwhitespace=false,         
  breaklines=true,                 
  captionpos=b,                    
  keepspaces=true,                 
  numbers=left,                    
  numbersep=5pt,                  
  showspaces=false,                
  showstringspaces=false,
  showtabs=false,                  
  tabsize=2
}

% images
\usepackage{graphicx}

\author{Kelly Christensen}
\title{Modélisation des transcriptions ALTO avec la TEI}

%%% DOCUMENT %%%
\begin{document}
	\begin{titlepage}
		\begin{center}
			
			\bigskip
			
			\begin{large}				
				ÉCOLE NATIONALE DES CHARTES\\
				UNIVERSITÉ PARIS, SCIENCES \& LETTRES
			\end{large}
			\begin{center}\rule{2cm}{0.02cm}\end{center}
			
			\bigskip
			\bigskip
			\bigskip
			\begin{Large}
				\textbf{Kelly Christensen}\\
			\end{Large}
		%selon le cas
			\begin{normalsize} \textit{licenciée ès enseignement musical}\\
				\textit{diplômée de master musicologie}\\
				\textit{diplômée de doctorat musicologie}
			\end{normalsize}
			
			\bigskip
			\bigskip
			\bigskip
			
			\begin{Huge}
				\textbf{Modélisation des transcriptions ALTO avec la TEI}\\
			\end{Huge}
			\bigskip
			\bigskip
			\begin{LARGE}
				\textbf{En complétant le pipeline du projet \textit{Gallic(orpor)a}}\\
			\end{LARGE}
			
			\bigskip
			\bigskip
			\bigskip
			\begin{large}
			\end{large}
			\vfill
			
			\begin{large}
				Mémoire 
				pour le diplôme de master \\
				\og{} Technologies numériques appliquées à l'histoire \fg{} \\
				\bigskip
				2022
			\end{large}
			
		\end{center}
	\end{titlepage}
	
	\thispagestyle{empty}	
	\cleardoublepage
	
	%%% FRONT MATTER %%%
	\frontmatter
	
	\chapter{Résumé}
	\subfile{sections/abstract/abstract.tex}
	
	\chapter{Remerciements}
	\lettrine{M}es remerciements vont tout d'abord à\dots
	
	%bibliographie ici
	%\printbibliography
	
	\chapter{Introduction}
	\subfile{sections/intro/introduction.tex}
	
	\thispagestyle{empty}
	\cleardoublepage
	
	%%% MAIN MATTER %%%
	\mainmatter
	
	%%% PARTIE I. %%%
	\part{Présentation du projet}
	
	\chapter{Qu'est-ce que l'HTR ?}
	\subfile{sections/chap1/chap1.tex}
	
	\chapter{Au commencement, il y avait les \textit{guidelines SegmOnto}}
	\subfile{sections/chap2/chap2.tex}

	\chapter{Le rêve du projet \textit{Gallic(orpor)a}}
	\subfile{sections/chap3/chap3.tex}
	
	
	%%% PARTIE II. %%%
	\part{Exposition de la préparation et du travail d'analyse}

	\chapter{Un pipeline visant à tout rassembler}
	\subfile{sections/chap4/chap4.tex}
	
	\chapter{L'analyse des structures des données XML}
	\subfile{sections/chap5/chap5.tex}
	
	\chapter{À la recherche des métadonnées}
	\subfile{sections/chap6/chap6.tex}
	

	%%% PARTIE III. %%%
	\part{Mise en opérationnelle du projet}
	
	\chapter{La génération du \texttt{<teiHeader>}}
	\subfile{sections/chap7/chap7.tex}
	
	\chapter{La modélisation de la \texttt{<sourceDoc>}}
	\subfile{sections/chap8/chap8.tex}
	
	\chapter{Les données textuelles produites}
	\subfile{sections/chap9/chap9.tex}
	
	\chapter*{Conclusion}
	\subfile{sections/conclusion/conclusion.tex}
	
	\addcontentsline{toc}{chapter}{Conclusion}
	
	%les annexes
	\appendix
	\chapter{Données}
	\subfile{sections/data/data.tex}
	
	\backmatter

% index à mettre ici si index	
%	\printindex

%glossaire si glossaire
	\printglossaries

% si figures
	\listoffigures

%si tables
	\listoftables

	\tableofcontents
	
\end{document}