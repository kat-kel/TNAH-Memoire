\documentclass[class=article, crop=false]{standalone}
\usepackage[subpreambles=true]{standalone}
\usepackage{import}
\usepackage{blindtext}
\begin{document}

\medskip

Quand des modèles \acrshort{ocr} (\textit{Optical Character Recognition}) et \acrshort{htr} (\textit{Handwriting Text Recognition}) extraient les données d'une ressource textuelle numérisée, les informations relatives à la structure physique de l'image risquent de se perdre. Un schéma \acrshort{xml} standardisé qui s'appelle \acrshort{alto} (\textit{Analyzed Layout and Text Object}) a été créé afin de conserver et structurer ces données non-textuelles et géometriques en les tenant en relation avec le contenu textuel. La plupart des modèles \acrshort{ocr} et \acrshort{htr} compte sur ce schéma. Cependant \acrshort{alto} ne convient pas bien à l'édition numérique ni aux traitements automatique du langage. Les éditeurs et les chercheurs en lettres attendent un schéma \acrshort{xml} plus courant dans le monde des humanités numériques : la \acrshort{tei} (\textit{Text Encoding Initiative}). Il faut donc un mapping pour transformer un fichier \acrshort{alto} en fichier \acrshort{tei} sans perdre aucune donnée lors du processus. Cette transformation automatisée permet à conserver les données particulières au schéma \acrshort{alto}, telles que celles sur la segmentation et sur la structure physique du document numérisé, ainsi qu'à exploiter le contenu textuel de la ressource textuelle. La flexibilité de la \acrshort{tei} et son usage très répandu rendent le schéma idéal pour mieux valoriser les données produites par les modèles \acrshort{ocr} et \acrshort{htr}.\\

Dans le cadre du stage pour obtenir le diplôme de Master 2 \og Technologies numériques appliquées à l'histoire \fg{}, ce mémoire porte sur la modélisation de la transformation de \acrshort{alto} en \acrshort{tei}. Cette modélisation a été réalisée dans le cadre du projet \textit{Gallic(orpor)a}, financé par la \acrshort{bnf} (Bibliothèque nationale de France) lors d'un stage qui a eu lieu au sein du laboratoire \acrshort{almanach} (Automatic Language Modelling and Analysis \& Computational Humanities) entre avril et juillet 2022.\\
	
	\textbf{Mots-clés~:} HTR, OCR, ALTO, TEI, TAL, édition numérique.
	
	\textbf{Informations bibliographiques~:} Kelly Christensen, \textit{Modélisation des transcriptions ALTO avec la TEI. En complétant le pipeline du projet} Gallic(orpor)a, mémoire de master \og{}Technologies numériques appliquées à l'histoire\fg{}, dir. [Noms des directeurs], École nationale des chartes, 2022.
	
\end{document}
\documentclass[../main.tex]{subfiles}