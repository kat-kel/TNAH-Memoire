\documentclass[class=article, crop=false]{standalone}
\usepackage[subpreambles=true]{standalone}
\usepackage{import}
\usepackage{blindtext}

%%%%%%%%%%%%%%%%%%%%%%%%
%			GLOSSAIRE
\usepackage[acronym]{glossaries}
\makeglossaries
\newglossaryentry{htr}
{
    name=Handwritten Text Recognition,
    description={La reconnaissance du texte écrit sur une image numérique}
}
\newacronym{HTR}{HTR}{Handwritten Text Recognition}

\newglossaryentry{ocr}
{
    name=Optical Character Recognition,
    description={La reconnaissance des polices du texte sur une image numérique}
}
\newacronym{OCR}{OCR}{Optical Character Recognition}

\newglossaryentry{TEI}
{
	name={Text Encoding Initiative},
	description={Normes internationales de l'encodage des documents texts}
}
\newacronym{tei}{TEI}{Text Encoding Initiative}

\newglossaryentry{iiif}
{
	name={International Image Interoperability Framework},
	description={Normes internationales de l'exploitation des images numériques et de leurs métadonnées par API}
}
\newacronym{IIIF}{IIIF}{International Image Interoperability Framework}

\newacronym{ALTO}{ALTO}{Analyzed Layout and Text Object}

\newacronym{XML}{XML}{eXtensible Markup Language}

\newacronym{BNF}{BnF}{Bibliothèque nationale de France}

\begin{document}

\medskip

Quand des modèles \acrshort{OCR} et \acrshort{HTR} extraient les données d'une ressource textuelle numérisée, les informations relatives à la structure physique de l'image risquent de se perdre. Un schéma XML standardisé qui s'appelle \acrshort{ALTO} a été créé afin de conserver et structurer ces données non-textuelles et géometriques en les tenant en relation avec le contenu textuel. La plupart des modèles \acrshort{OCR} et \acrshort{HTR} compte sur ce schéma. Cependant \acrshort{ALTO} ne convient pas bien à l'édition numérique ni aux traitements automatique du langage. Les éditeurs et les chercheurs en lettres attendent un schéma XML plus courant dans le monde des humanités numériques : la \acrshort{TEI}. Il faut donc un mapping pour transformer un fichier \acrshort{ALTO} en fichier \acrshort{TEI} sans perdre aucune donnée lors du processus. Cette transformation automatisée permet à conserver les données particulières au schéma \acrshort{ALTO}, telles que celles sur la segmentation et sur la structure physique du document numérisé, ainsi qu'à exploiter le contenu textuel de la ressource textuelle. La flexibilité de la \acrshort{TEI} et son usage très répandu rendent le schéma idéal pour mieux valoriser les données produites par les modèles \acrshort{OCR} et \acrshort{HTR}.\\

Dans le cadre du stage pour obtenir le diplôme de Master 2 \og~Technologies numériques appliquées à l'histoire~\fg{}, ce mémoire porte sur la modélisation de la transformation de \acrshort{ALTO} en \acrshort{TEI}. Cette modélisation a été réalisée dans le cadre du projet \textit{Gallic(orpor)a}, financé par la \acrshort{BNF} lors d'un stage qui a eu lieu au sein du laboratoire Automatic Language Modelling and Analysis \& Computational Humanities entre avril et juillet 2022.\\
	
	\textbf{Mots-clés~:} HTR, OCR, ALTO, TEI, TAL, édition numérique.
	
	\textbf{Informations bibliographiques~:} Kelly Christensen, \textit{D’ALTO à TEI, modélisation de transcriptions automatiques pour une pré-éditorialisant des textes}, mémoire de master \og{}Technologies numériques appliquées à l'histoire\fg{}, dir. Ségolène Albouy, École nationale des chartes, 2022.
	
\end{document}
\documentclass[../main.tex]{subfiles}