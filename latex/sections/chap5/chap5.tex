\documentclass[class=article, crop=false]{standalone}
\usepackage[subpreambles=true]{standalone}
\usepackage{import}
\usepackage{blindtext}
\begin{document}


\section{XML-ALTO}

\subsection{Qu'est-ce qu'est le format ALTO ?}

Décrire la création, le suivi, et l'objectif du format XML ALTO : enregistrer les infos sur la structure d'une image segmentée. 

\subsection{La structure des fichiers XML-ALTO}

Montrer la structure des données d'un fichier ALTO dans les deux formats qui sortent de Kraken : (1) ligne de texte encodé dans la balise \texttt{<TextLine>}, qui sort de Kraken via l'interface d'eScriptorium, et (2) ligne de texte encodé au niveau du glyph, qui sort directement de la ligne de commande de Kraken.

\section{XMl-TEI}

\subsection{Qu'est-ce qu'est la TEI ?}

Décrire la création, le suivi, et l'objectif de la TEI.

\subsection{Les éléments de base de la TEI}

Expliquer qu'il y a deux éléments essentiels de la racine, le \texttt{<teiHeader>} et le \texttt{<body>}. Ensuite expliquer l'utilité de l'élément facultatif \texttt{<sourceDoc>} et expliquer pourquoi il convient bien aux données de structure d'un fichier ALTO. 


\end{document}
\documentclass[../main.tex]{subfiles}