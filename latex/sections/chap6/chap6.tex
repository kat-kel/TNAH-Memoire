% PREAMBULE

% !BIB TS-program = biber
% !TEX TS-program = xelatexmk
% ITEX TS-program = latex

% !TEX spellcheck = French

\documentclass[class=article, crop=false]{standalone}
\usepackage[subpreambles=true]{standalone}
\usepackage{import}
\usepackage{blindtext}
\usepackage{fontspec}
\usepackage[french]{babel}
\usepackage{caption}
\usepackage{subcaption}
\usepackage{csquotes}
\usepackage{url}

%%%%%%%%%%%%%%%%%%%%%%%%
%			REFERENCES
% le package hyperref avec des options, si en local
\usepackage[pdfusetitle, pdfsubject ={Mémoire TNAH}, pdfkeywords={les mots-clés}]{hyperref}
\usepackage[backend=bibtex, sorting=nyt, style=verbose-ibid]{biblatex}
\addbibresource{../../../bib.bib}

%%%%%%%%%%%%%%%%%%%%%%%%
%			GLOSSAIRE
\usepackage[acronym]{glossaries}
\newglossaryentry{htr}
{
    name=Handwritten Text Recognition,
    description={La reconnaissance du texte écrit sur une image numérique}
}
\newacronym{HTR}{HTR}{Handwritten Text Recognition}

\newglossaryentry{ocr}
{
    name=Optical Character Recognition,
    description={La reconnaissance des polices du texte sur une image numérique}
}
\newacronym{OCR}{OCR}{Optical Character Recognition}

\newglossaryentry{Inria}
{
    name=Inria,
    description={Institut national de recherche en sciences et technologies du numérique}
}
\newacronym{INRIA}{Inria}{Institut national de recherche en sciences et technologies du numérique}

\newacronym{almanach}{ALMAnaCH}{Automatic Language Modelling and Analysis \& Computational Humanities}

\newglossaryentry{enc}
{
    name=École nationale des chartes,
    description={Grande école bla bla bla}
}
\newacronym{ENC}{ENC}{École nationale des chartes}

\newglossaryentry{HTR-United}
{
    name=HTR-United,
    description={HTR-United is a catalog and an ecosystem for sharing and finding ground truth for optical character or handwritten text recognition (OCR/HTR)}
}

\newglossaryentry{CLab}
{
	name=CREMMALab,
	description={Consortium pour la reconnaissance
d’'écriture manuscrite des matériaux anciens}
}
\newacronym{CREMMA}{CREMMA}{Consortium Reconnaissance
d’Écriture Manuscrite des Matériaux Anciens}

\newglossaryentry{tei}
{
	name={Text Encoding Initiative},
	description={Normes internationales de l'encodage des documents textes}
}
\newacronym{TEI}{TEI}{Text Encoding Initiative}

\newglossaryentry{iiif}
{
	name={International Image Interoperability Framework},
	description={Normes internationales de l'exploitation des images numériques et de leurs métadonnées par API}
}
\newacronym{IIIF}{IIIF}{International Image Interoperability Framework}

\newacronym{ALTO}{ALTO}{Analyzed Layout and Text Object}

\newacronym{XML}{XML}{eXtensible Markup Language}

\newacronym{BNF}{BnF}{Bibliothèque nationale de France}

\newacronym{RDF}{RDF}{Resource Description Framework}

\newacronym{TAL}{TAL}{Traitement automatique des langues}

\newacronym{ARK}{ARK}{Archival Resource Key}

\newacronym{DTS}{DTS}{Distributed Text Services}

\newglossaryentry{iiifapi}
{
	name={IIIF Image API},
	description={Un service de web qui renvoie une image suite à une requête standardisée HTTP(S). L'URI peut préciser la région, la taille, la rotation, la qualité, les caractéristiques, et le format de l'image demandée.}
}
\newacronym{API}{API}{Application Programming Interface}

\newglossaryentry{odd}
{
	name={One Document Does it all},
	description={Un fichier XML TEI qui précise les règles d'un schème TEI personnalisé.}
}
\newacronym{ODD}{ODD}{One Document Does it all}

\newacronym{JSON}{JSON}{JavaScript Object Notation}

\newacronym{HTML}{HTML}{HyperText Markup Language}

\newacronym{METS}{METS}{Metadata Encoding and Transmission Standard}

\newacronym{YAML}{YAML}{Yet Another Markup Language}

\newacronym{SRU}{SRU}{Search/Retrieve via URL}

\newglossaryentry{unimarc}
{
	name={UNIMARC},
	description={Une référence pour l’échange de données en format XML}
}

\newacronym{SUDOC}{SUDOC}{Système Universitaire de Documentation}

%%%%%%%%%%%%%%%%%%%%%%%%
%			DIAGRAM
\usepackage{tikz}
\usetikzlibrary{positioning}
\usetikzlibrary{calc, matrix, shapes.geometric, arrows}
\usepackage{pgfplots}
\usepackage{array}
\usepackage{tabularx}


%%%%%%%%%%%%%%%%%%%%%%%%
%			CODE
\usepackage{listings}
\usepackage{color}
\definecolor{codegray}{rgb}{0.5,0.5,0.5}
\definecolor{codepurple}{rgb}{0.58,0,0.82}
\definecolor{cyan}{rgb}{0.0,0.6,0.6}
\definecolor{codegreen}{rgb}{0,0.6,0}
\definecolor{backcolour}{rgb}{0.95,0.95,0.92}

\lstdefinelanguage{XML}{
  backgroundcolor=\color{backcolour},  
  basicstyle=\ttfamily\footnotesize,
  morestring=[s]{"}{"},
  moredelim=[s][\color{black}]{>}{<},
  morecomment=[s]{!--}{--},
  commentstyle=\color{codegreen},
  moredelim=[s][\color{red}]{\ }{=},
  stringstyle=\color{blue},
  identifierstyle=\color{cyan},
  numberstyle=\tiny\color{codegray},
  breakatwhitespace=false,         
    breaklines=true,                 
    captionpos=b,                    
    keepspaces=true,                 
    numbers=left,                    
    numbersep=5pt,                  
    showspaces=false,                
    showstringspaces=false,
    showtabs=false,                  
    tabsize=2
}

%Code listing style named "json"
\lstdefinestyle{json}{
  backgroundcolor=\color{backcolour}, 
  basicstyle=\ttfamily\footnotesize,
  commentstyle=\color{codegreen},
  numberstyle=\tiny\color{codegray},
  basicstyle=\ttfamily\footnotesize,
  breakatwhitespace=false,         
  breaklines=true,                 
  captionpos=b,                    
  keepspaces=true,                 
  numbers=left,                    
  numbersep=5pt,                  
  showspaces=false,                
  showstringspaces=false,
  showtabs=false,                  
  tabsize=2
}

%%%%%%%%%%%%%%%%%%%%%%%%
%%%%%%%%%%%%%%%%%%%%%%%%
%			DOCUMENT
%%%%%%%%%%%%%%%%%%%%%%%%
%%%%%%%%%%%%%%%%%%%%%%%%
\begin{document}
Tout document \acrshort{TEI} doit comporter des métadonnées qui renseignent sur le document \acrshort{TEI} lui-même ainsi que sur le texte qu'il représente. Dans le cadre du projet \textit{Gallic(orpor)a}, le texte produit est toujours la transcription d'un document source numérisé. Les métadonnées du fichier doivent en plus de la ressource numérisée prendre en compte le document source physique qui a été numérisé. Pour les besoins du projet \textit{Gallic(orpor)a}, les métadonnées du document \acrshort{TEI} portent sur les trois objets de texte suivants~:
\begin{enumerate}
\item la ressource numérique, c'est-à-dire les informations sur le document \acrshort{TEI} produit par le pipeline \textit{Gallic(orpor)a}
\item la numérisation du document source, c'est-à-dire le fac-similé numérique stocké dans la base de données Gallica
\item le document source physique, qui a été numérisé
\end{enumerate}

Malgré la diversité documentaire du projet \textit{Gallic(orpor)a}, le \texttt{<teiHeader>} de tout document \acrshort{TEI} que le pipeline produit est toujours constitué d'informations sur ces trois objets. Chacun porte sur un fac-similé numérique, dérivé d'un document source physique et transformé en objet numérique par le pipeline. Les métadonnées doivent être les mêmes pour tous les documents issus du pipeline. Un manuscrit écrit par plusieurs mains et sans éditeur, issu d'un scriptorium à une date approximative, doit disposer des mêmes types de métadonnées qu'un imprimé écrit par une autrice et publié par un éditeur. Une solution pour surmonter ce défi est de réduire le nombre de métadonnées à l'essentiel de sorte à toujours respecter la même structure au sein de la collection.

Selon les normes de la \acrshort{TEI}, trois parties peuvent constituer le \texttt{<teiHeader>}~: une description bibliographique de l'encodage (\texttt{<fileDesc>}), une description des aspects non bibliographiques tels qu'un classement du contenu textuel qui appartiennent au texte représenté (\texttt{<profileDesc>}), et une description technique de l'encodage  dans \texttt{<encodingDesc>}. L'objectif de ce chapitre est d'expliquer la structure du \texttt{<teiHeader>} que nous avons modélisée dans le cadre du projet \textit{Gallic(orpor)a}. Le chapitre~\ref{chap:header} explique comment réaliser cette structure en récupérant toutes les métadonnées déterminées essentielles, selon la modélisation exposée dans ce chapitre.


%%%%%%%%%%%%%%%%%%%%%%%%
%			description bibliographique
\section{La description bibliographique (\texttt{<fileDesc>})}
Dans un premier temps, la ressource numérique elle-même doit être décrite. Ces informations sont organisées dans l'élément \texttt{<fileDesc>}. Traduit littéralement en français comme \textit{la description du fichier}, le \texttt{<fileDesc>} décrit le document \acrshort{TEI} lui-même. Cette description doit porter sur les trois aspects suivants~:
\begin{enumerate}
\item le titre et la responsabilité de la ressource numérique (\texttt{<titleStmt>})
\item la distribution de la ressource, y compris les droits d'utilisation (\texttt{<publicationStmt>})
\item le document source (\texttt{<sourceDesc>})
\end{enumerate}
D'autres éléments peuvent être ajoutés dans le \texttt{<teiHeader>} afin d'apporter encore plus de détails bibliographiques. Par exemple, le \texttt{<editionStmt>} précise l'édition de l'œuvre. Les documents \acrshort{TEI} produits par le pipeline \textit{Gallic(orpor)a} ne profitent pas de cet élément parce que certains documents du corpus traité, tels que les manuscrits, n'ont pas d'édition et on veut que chaque document \acrshort{TEI} ait les mêmes types de métadonnées dans le \texttt{<teiHeader>}. Au contraire, d'autres éléments facultatifs, tel que l'élément \texttt{<extent>}, sont produits par le pipeline. L'élément \texttt{<extent>} est utile parce qu'il porte sur la taille de la ressource (nombre de pages transcrites) ce qui permet d'indiquer le volume de total de la ressource.

\subsection{Le titre et la responsabilité (\texttt{<titleStmt>})}
Après le titre (\texttt{<title>}), il est recommandé d'indiquer l'individu ou les individus responsables de la création du texte représenté et/ou de la ressource numérique. Dans le cadre du projet \textit{Gallic(orpor)a}, nous avons conçu un schéma \texttt{<titleStmt>} simple qui comporte : titre, auteur(s), et responsables de la production de la ressource numérique.

\subsubsection{La responsabilité}
Entre les lignes 4 et 21 de la Figure~\ref{fig:titleStmt1}, l'élément \texttt{<respStmt>} contient des informations sur l'équipe du projet \textit{Gallic(orpor)a}. Tout document encodé par le pipeline comporte un titre (\texttt{<title>}), le nom de la  personne ou des personnes auxquels est attribuée la responsabilité du texte (\texttt{<author>}), et la mention de l'équipe qui a conçu le pipeline et l'application \texttt{alto2tei} (\texttt{<respStmt>}). La déclaration de responsabilité peut être personnalisée selon le projet ou selon l'équipe qui utilise le pipeline \textit{Gallic(orpor)a} ou l'application \texttt{alto2tei} pour créer la ressource numérique. En général, elle devrait contenir une phrase qui résume la nature de la création de la ressource, telle que la phrase \og{}Transformation from ALTO4 to TEI by\fg{}, balisée dans l'élément \texttt{<resp>} (ligne 5, Fig.~\ref{fig:titleStmt1}). Ensuite, elle devrait contenir des éléments \texttt{<persName>} pour renseigner le nom des personnes en charge de la production du document.

\begin{figure}[htp]
\centering
\begin{lstlisting}[language=XML]
<titleStmt>
	<title>Titre du document source traité</title>
	<author>Auteur</author>
	<respStmt>
		<resp>Transformation from ALTO4 to TEI by</resp>
		<persName>
			<forename>Kelly</forename>
			<surname>Christensen</surname>
			<ptr type="orcid" target="000000027236874X"/>
		</persName>
		<persName>
			<forename>Simon</forename>
			<surname>Gabay</surname>
			<ptr type="orcid" target="0000000190944475"/>
		</persName>
		<persName>
			<forename>Ariane</forename>
			<surname>Pinche</surname>
			<ptr type="orcid" target="0000000278435050"/>
		</persName>
	</respStmt>
</titleStmt>
\end{lstlisting}
\caption{Les informations sur le titre de la ressource}
\label{fig:titleStmt1}
\end{figure}

\subsubsection{Le titre}
La ressource produite par le pipeline \textit{Gallic(orpor)a} a besoin de son propre titre, et l'application \texttt{alto2tei} que j'ai créée lui attribue le nom du document source. Le schéma \acrshort{TEI} permet de construire un nouveau titre lors de l'encodage ou d'associer plusieurs titres au document \acrshort{TEI}. Cependant, nous avons décidé d'utiliser le titre du document source fourni par les métadonnées produites par la BnF, au lieu de construire un nouveau titre.

Dans le schéma \acrshort{TEI}, plusieurs titres peuvent être indiqués dans le \texttt{<titleStmt>}. Par exemple, le projet \textit{The Bodelian First Folio} a encodé les premières éditions des drames de Shakespeare en \acrshort{TEI} et chacun porte plusieurs types de titre\footcite{BodelianFirstFolio}. L'encodage de la comédie \textit{Twelfth Night} possède un titre du type ``\textit{statement}'' qui représente le titre tel qu'il se trouve sur l'imprimé historique (ligne 2, Fig.~\ref{fig:printTitle}). Il donne aussi une variante du titre qui est le nom du recueil dans lequel se trouve la pièce (ligne 3, Fig.~\ref{fig:printTitle}). 
Enfin, l'encodage présente le titre qui sert à identifier la source aux archives, \og{}Bodelian First Folio, Arch. G c.7\fg{} (ligne 4, Fig.~\ref{fig:printTitle})

\begin{figure}[ht]
\centering
\begin{lstlisting}[language=XML]
<titleStmt>
	<title type="statement">Twelfe Night, or What You Will from Mr. William Shakespeares comedies, histories, &amp; tragedies. Published according to the true originall copies.</title>
	<title type="variant">Mr. VVilliam Shakespeares comedies, histories, &amp; tragedies</title>
	<title type="distinctive">Bodleian First Folio, Arch. G c.7</title>
<!-- ... -->
</titleStmt>
\end{lstlisting}
\caption{Les informations sur le titre d'un imprimé \protect\footnotemark}
\label{fig:printTitle}
\end{figure}
\footnotetext{\cite{BodelianFirstFolio}}

Pour un manuscrit, l'attribution d'un titre pourrait obliger la création d'un nom qui ne se trouve pas sur le document source. Par exemple, le projet \textit{CatCor} qui a encodé des lettres écrites par Catherine II de la Russie a choisi d'attribuer un titre qui s'appuie sur l'identifiant du document source. Dans l'encodage d'une lettre destinée à Fréderick II le 21 juillet 1744, le \texttt{<title>} dans le \texttt{<titleStmt>} est un titre qui n'apparaît nulle parte sur la source (ligne 2, Fig.~\ref{fig:manuTitle})\footcite{catcorprojectLetter02633Frederick2021}. Contrairement à l'encodage de l'imprimé de Shakespeare, l'encodage de la lettre manuscrite ne donne pas de type au titre attribué à la ressource \acrshort{TEI}. La classification \texttt{@type} du \texttt{<title>} n'est pas exigée par le schéma \acrshort{TEI}, mais elle est recommandée s'il y a plusieurs titres. Puisque la lettre manuscrite n'a pas de titre donné par l'auteur, l'équipe de \textit{CatCor} l'en a créé un et uniquement un pour l'encodage de la lettre manuscrite.

\begin{figure}[ht]
\centering
\begin{lstlisting}[language=XML]
<titleStmt>
	<title>CatCor Project: letter-02633</title>
<!-- ... -->
</titleStmt>
\end{lstlisting}
\caption{Les informations sur le titre d'un manuscrit \protect\footnotemark}
\label{fig:manuTitle}
\end{figure}
\footnotetext{\cite{catcorprojectLetter02633Frederick2021}}

Afin d'encoder les éléments \texttt{<title>} à l'échelle, il faut qu'un logiciel (1) ait d'accès aux métadonnées et (2) sache la nature des titres associés au document source. Certains corpus auront d'accès à des métadonnées déjà classifiées. Le catalogue général de la \acrshort{BNF}, par exemple, organise ses métadonnées dans une structure de données \acrshort{XML} \Gls{unimarc}. Chaque titre associé au document est balisé dans des éléments \acrshort{XML} qui portent des noms ou des attributs qui précisent la nature du titre. Par exemple, l'\Gls{unimarc} présente le type \og{}titre uniforme\fg{} dans l'élément \texttt{<mxc:datafield tag="500">} et le type \og{}titre de forme\fg{} dans l'élément \texttt{<mxc:datafield tag="503">}. En récupérant les métadonnées depuis une source ainsi organisée, un logiciel pourrait attribuer un type à l'élément \texttt{<title>}. 

\subsubsection{L'auteur}
Le \texttt{<titleStmt>} renseigne sur l'individu ou les individus aux lesquels la propriété intellectuelle du document source est attribuée. Cette donnée est encodée dans l'élément \texttt{<author>}. L'encodage de la responsabilité peut être minimal et ne contenir qu'une chaine de caractères avec le nom des auteurs ou bien un simple l'élément \texttt{<name>}. Voir Figure~\ref{fig:authorsimple}. Sinon, la description peut être enrichie par d'autres éléments afin d'apporter plus de détail sur les composants du nom de l'auteur. Voir Figure~\ref{fig:authorcomplex}.

\begin{figure}[ht]
\centering
\begin{lstlisting}[language=XML]
<author>
	<name>Donatien Alphonse François de Sade</name>
</author>
\end{lstlisting}
\caption{L'auteur simple}
\label{fig:authorsimple}
\end{figure}

\begin{figure}[ht]
\centering
\begin{lstlisting}[language=XML]
<author xmlid="Sa1">
	<persName>
		<forename>Donatien Alphonse François</forename>
		<nameLink>de</nameLink>
		<surname>Sade</surname>
		<ptr type="isni" target="0000000084961458"/>
	</persName>
</author>
\end{lstlisting}
\caption{L'auteur enrichi}
\label{fig:authorcomplex}
\end{figure}

\noindent L'élément \texttt{<ptr>} qui veut dire \textit{pointer} en anglais indique une ressource ou une donnée externe, tel que l'identifiant ISNI, afin d'enrichir les informations de l'objet auquel il est attaché. On voit un exemple du \textit{pointer} sur les lignes 9, 14, et 19 de la Figure~\ref{fig:titleStmt1} où l'élément indique l'ORCID unique de l'individu responsable de la création de la ressource numérique.

L'exemple d'un document \acrshort{TEI} produit par le pipeline \textit{Gallic(orpor)a} et donc de notre modélisation du \texttt{<author>} se voit dans la Figure~\ref{fig:authorcomplex}. Le nom de l'auteur est divisé en trois composants, selon les données \Gls{unimarc} fournies par le catalogue général de la \acrshort{BNF}. Le catalogue désigne \textit{Donatien Alphonse François de} comme la \og{}partie du nom autre que l’élément d’entrée\fg{}. L'\Gls{unimarc} balise cette partie secondaire dans l'élément \texttt{<mxc:subfield code="b">} de l'élément \texttt{<mxc:datafield tag="700">} ou \texttt{<mxc:datafield tag="701">}\footcite{ManuelUNIMARCFormat}. Mon application \texttt{alto2tei}, développée pour le pipeline \textit{Gallic(orpor)a}, ensuite tire la partie \textit{de} du nom puisqu'elle est un lien entre ses composants et peut donc être encodée dans l'élément \texttt{<nameLink>}. Le catalogue général catégorise le nom \textit{Sade} comme le nom d'entrée de l'auteur, que l'\Gls{unimarc} balise dans l'élément \texttt{<mxc:subfield code="a">}. Notre modélisation balise donc cette donnée dans l'élément \texttt{<surname>}, c'est-à-dire le nom de famille. Dans le schéma \acrshort{TEI}, le \texttt{<titleStmt>} peut renseigner plusieurs auteurs en répétant l'élément \texttt{<author>} comme dans la Figure~\ref{fig:multipleauthors}.

Le schéma \acrshort{TEI} permet d'indiquer le rôle de chacun des auteurs listés dans un \texttt{<titleStmt>}. Toutefois, par manque de métadonnées disponibles dans les sources de données que nous avons ciblées, notre modélisation actuelle ne s'appuie pas sur cette donnée. Dans l'exemple de la Figure~\ref{fig:multipleauthors}, les données \Gls{unimarc} récupérées du catalogue général de la \acrshort{BNF} n'indiquent pas que Giacomo Meyerbeer est le compositeur de l'opéra \textit{Les Huguenots} ni que lui et le librettiste Eugène Scribe partagent en parts égales la responsabilité pour l'œuvre.

\begin{figure}[ht]
\centering
\begin{lstlisting}[language=XML]
<titleStmt>
	<title>Les Huguenots</title>
	<author xml:id="Me1">
		<persName>
			<forename>Giacomo</forename>
			<surname>Meyerbeer</surname>
			<ptr type="isni" target="0000000122817116"/>
		</persName>
	</author>
	<author xml:id="Sc1">
		<persName>
			<forename>Eugène</forename>
			<surname>Scribe</surname>
			<ptr type="isni" target="000000012122970X"/>
		</persName>
	</author>
	<author xml:id="De1">
		<persName>
			<forename>Émile</forename>
			<surname>Deschamps</surname>
			<ptr type="isni" target="0000000122807567"/>
		</persName>
	</author>
	<author xml:id="Ro1">
		<persName>
			<forename>Gaetano</forename>
			<surname>Rossi</surname>
			<ptr type="isni" target="0000000121219499"/>
		</persName>
	</author>
<!-- ... -->
</titleStmt>
\end{lstlisting}
\caption{Plusieurs auteurs dans un \texttt{<titleStmt>}}
\label{fig:multipleauthors}
\end{figure}


\subsection{La taille de la ressource numérique (\texttt{<extent>})}
Lors de la phase de reconnaissance du texte, les modèles \acrshort{HTR} génèrent un certain nombre de fichiers \acrshort{XML} \acrshort{ALTO}. Grâce à l'unité indiquée comme \og{}images\fg{} dans l'élément \texttt{<measure>}, la taille est facilement calculé à partir du compte des fichiers \acrshort{XML} \acrshort{ALTO}, les mêmes fichiers produits et ensuite traités pour constituer l'élément \texttt{<sourceDoc>} de la ressource numérique. 
\begin{figure}[htp]
\centering
\begin{lstlisting}[language=XML]
<extent>
	<measure unit="images" n="20"/>
</extent>
\end{lstlisting}
\caption{La taille de la ressource}
\label{fig:extent}
\end{figure}

\subsection{La distribution de la ressource numérique (\texttt{<publicationStmt>})}
L'élément \texttt{<publicationStmt>} contient des données importantes qui renseignent sur la distribution et les droits d'utilisation de la ressource numérique en format \acrshort{XML} \acrshort{TEI}. Toutes les métadonnées présentes dans cet élément  portent sur le contexte de la création de la ressource, aucune sur le document source représenté. Pour toutes les ressources produites par le pipeline, les données du \texttt{<publicationStmt>} ne doivent pas changer, à l'exception de la date de création de la ressource, que l'application \texttt{alto2tei} génère automatiquement.

Comme montre la Figure~\ref{fig:pubStmt}, trois données du \texttt{<publicationStmt>} peuvent être personnalisées. L'entité reconnue comme l'éditeur (\textit{publisher}) de la ressource peut être changé selon le projet. Dans l'exemple de la Figure~\ref{fig:pubStmt}, le \textit{publisher} est \og{}Gallic(orpor)a\fg{}. L'autorité qui l'a financé et qui est civilement responsable pour la ressource est le DataLab de la \acrshort{BNF} indiqué dans l'élément \texttt{<authority>}. Enfin, les droits d'utilisation de la ressource sont indiqués par les éléments \texttt{<availability>} et \texttt{<licence>}.

\begin{figure}[htp]
\centering
\begin{lstlisting}[language=XML]
<publicationStmt>
	<publisher>Gallic(orpor)a</publisher>
	<authority>BnF DATALab</authority>
	<availability status="restricted" n="cc-by">
		<licence target="https://creativecommons.org/licenses/by/4.0/"/>
	</availability>
	<date when="2022-07-29"/>
</publicationStmt>
\end{lstlisting}
\caption{Plusieurs auteurs dans un \texttt{<publicationStmt>}}
\label{fig:pubStmt}
\end{figure}


\subsection{Le document source (\texttt{<sourceDesc>})}
Le dernier aspect de la description bibliographique du \texttt{<teiHeader>} porte sur le document source physique dont le texte est représenté. Cet aspect est balisé dans l'élément \texttt{<sourceDesc>} qui est le dernier élément du \texttt{<fileDesc>} dans notre modélisation \acrshort{TEI}. Cet élément s'appuie sur un grand nombre de sources externes afin de renseigner cette partie de la manière la plus diversifiée et précise possible. La récupération de ces données est expliquée dans le chapitre~\ref{chap:header}.

\subsubsection{La citation bibliographique (\texttt{<bibl>})}

Dans un premier temps, la description de la source représentée dans le \texttt{<sourceDesc>} s'appuie sur les informations bibliographiques. En répétant certaines informations du \texttt{<titleStmt>}, l'élément  \texttt{<bibl>} présente (1) les individus auxquels est attribuée la propriété intellectuelle du document, (2) le titre du document source--qui est le même titre attribué à la ressource numérique produite, selon notre modélisation, (3) son lieu de publication, (4) l'éditeur, et (5) la date de publication. Voir l'arborescence du \texttt{<bibl>} dans la Figure~\ref{fig:bibl}.

%Pas clair
%Puisque la notice du catalogue pour le document source montré dans l'exemple de la Figure~\ref{fig:bibl} peut être %trouvée par l'application \texttt{alto2tei}, les données du \texttt{<bibl>} sont plus complète qu'elle auraient été si l%'application avait besoin de compter uniquement sur le \textit{manifest} \acrshort{IIIF}. 

Les dates de publication ou d'apparition du document source physique sont représentées dans l'élément \texttt{<date>}. Le schéma \acrshort{TEI} permet d'ajouter plusieurs détails à la date. Par exemple, notre modélisation \acrshort{TEI} attribue un degré de certitude à la date attribué à la source quand la ressource s'appuie sur le catalogue général de la \acrshort{BNF}. Les données \Gls{unimarc} du catalogue nous permet de déterminer un degré de certitude quant à la date parmi les trois valeurs suivantes~: \og{}low\fg{}, \og{}medium\fg{}, ou \og{}high\fg{}. De plus, la \acrshort{TEI} recommande fortement l'encodage de la date dans un format standardisé. Notre modélisation \acrshort{TEI} conseille donc l'encodage d'une date selon le format visualisé dans la ligne 21 de la Figure~\ref{fig:bibl}, où l'attribut \texttt{@resp} représente l'autorité qui a accordé le degré de certitude.

L'exemple de la Figure~\ref{fig:bibl} est issu d'un imprimé du \textsc{xviii}\up{e} siècle dont la date de publication est connue mas pas l'éditeur. L'éditeur n'est indiqué dans les données \Gls{unimarc} de la \acrshort{BNF} comme \og{}s.n.\fg{} pour signaler que la bibliothèque n'en est pas certaine. Notre modélisation \acrshort{TEI} envisage à répondre aux données incertaines ou manquantes des sources de données en laissant vides ou incertaines les données intégrées dans l'arborescence du \texttt{<teiHeader>}. Par conséquent, l'élément \texttt{<publisher>} dans la Figure~\ref{fig:bibl} contient la donnée du catalogue même si elle indique une manque d'information.

Le \texttt{<bibl>} d'un manuscrit contient les mêmes éléments que celui d'un imprimé. Même si un manuscrit n'a pas d'éditeur ni n'est pas publié de la même manière qu'un imprimé, la structuration \acrshort{TEI} reste la même et l'élément \texttt{<bibl>} contiendra tout de même  les sous-éléments \texttt{<pubPlace>} et \texttt{<publisher>}. Normalement, le catalogue général de la \acrshort{BNF} n'indiquera pas de donnée pour ces aspects. Quand la donnée n'est pas disponible, notre modélisation exige que l'arborescence générique est toujours gardée et qu'elle indique dans l'élément standard que la donnée n'existe pas.

\begin{figure}[htp]
\centering
\begin{lstlisting}[language=XML]
<sourceDesc>
	<bibl>
		<ptr target="http://catalogue.bnf.fr/ark:/12148/cb30369299r"/>
		<author ref="#Re1">
			<persName>
				<forename>Jean-François</forename>
				<surname>Regnard</surname>
				<ptr type="isni" target="000000012118509X"/>
			</persName>
		</author>
		<author ref="#Du2">
			<persName>
				<forename>Charles</forename>
				<surname>Du Fresny</surname>
				<ptr type="isni" target="0000000140935001"/>
			</persName>
		</author>
		<title>Scènes françoises de la comédie italienne intitulée "la Foire S.-Germain" , comme elles ont paru dans les premières représentations</title>
		<pubPlace key="FR">Grenoble</pubPlace>
		<publisher>[s.n.]</publisher>
		<date when="1696" cert="high" resp="BNF">1696</date>
	</bibl>
	<msDesc>
<!-- ... -->
	</msDesc>
</sourceDesc>

\end{lstlisting}
\caption{La citation bibliographique (\texttt{<bibl>})}
\label{fig:bibl}
\end{figure}


\subsubsection{La description de la source physique (\texttt{<msDesc>})}
Le document source physique et son fac-similé numérique sont tous les deux décrits dans l'élément \texttt{<msDesc>}. Tandis que la citation bibliographique (\texttt{<bibl>}) porte sur le document en tant qu'œuvre, la \textit{description du manuscrit} (\texttt{<msDesc>}) porte sur le document en tant qu'un objet matériel dans le monde réel. Le \texttt{<msDesc>} a deux enfants principaux, voir Figure~\ref{fig:msDesc}. Premièrement, l'élément \texttt{<msIdentifier>} sert à identifier le document physique et le fac-similé numérique dans un catalogue. Deuxièmement, l'élément \texttt{<physDesc>} sert à décrire le  document manuscrit soit imprimé.

L'identification du document physique est très importante. Par exemple, un imprimé peut avoir plusieurs exemplaires~; chacun pourrait avoir des différences et une transcription distincte. Afin de bien indiquer quel objet a été traité par le pipeline \textit{Gallic(orpor)a}, il faut identifier le document source physique qui a servi de base aux prédictions des modèles \acrshort{HTR}. Les éléments \texttt{<idno>} dans le \texttt{<msDesc>} indiquent l'identifiant d'un document. L'identifiant du document source physique est l'élément principal descendant du \texttt{<msIdentifier>}.  Le \texttt{<idno>} alternatif est l'\acrshort{ARK} du fac-similé numérique sur Gallica. Le \texttt{<idno>} principal du \texttt{<msDesc>} est le cote du document physique selon le catalogue de la \acrlong{BNF}. Après cette description des sources numériques et physiques (\texttt{<msDesc>}), sur lesquels la ressource numérique \acrshort{TEI} est basée, la description du fichier \texttt{<fileDesc>} est complète selon notre modélisation.


\begin{figure}[htp]
\centering
\begin{lstlisting}[language=XML]
<sourceDesc>
	<bibl>
<!-- ... -->
	</bibl>
	<msDesc>
		<msIdentifier>
			<country key="FR"/>
			<settlement>Paris</settlement>
			<repository>Bibliothèque nationale de France</repository>
			<idno>YF-5877</idno>
			<altIdentifier>
				<idno type="ark">bpt6k1281160s</idno>
			</altIdentifier>
		</msIdentifier>
		<physDesc>
			<objectDesc>
				<p>Texte imprimé</p>
			</objectDesc>
		</physDesc>
	</msDesc>
</sourceDesc>

\end{lstlisting}
\caption{La description de la source (\texttt{<msDesc>})}
\label{fig:msDesc}
\end{figure}

\section{La description non bibliographique (\texttt{<profileDesc>})}
La description du fichier (\texttt{<fileDesc>}) porte sur les détails bibliographiques de la ressource numérique, tel que le document source. Ensuite, notre modélisation du \texttt{<teiHeader>} renseignent sur une description non bibliographique de la ressource numérique en format \acrshort{TEI}. Normalement la description non bibliographique informe sur les langues utilisées dans le texte. Elle peut aussi donner les noms de lieux ou de personnages référencés dans le texte. Notre modélisation est moins détaillé. Notre \texttt{<profileDesc>} renseigne simplement sur la langue du texte et sur l'application \texttt{alto2tei} qui a construit le document \acrshort{TEI}. Actuellement, cette application ne peut porter que sur une seule langue. C'est un point de modélisation des données qui pourrait évoluer pour, si besoins à l'avenir proposer un encodage capable d'accepter plusieurs langues du texte ou encore des index.

La Figure~\ref{fig:profileDesc} présente un \texttt{<profileDesc>} réalisé par l'application \texttt{alto2tei} et donc montre l'exemple de notre modélisation \acrshort{TEI}. La langue identifiée est français. Cette information a été récupérée depuis le \textit{manifest} \acrshort{IIIF} du fac-similé numérique de Gallica. Mais quand l'application \texttt{alto2tei} peut aussi accéder aux données du catalogue général de la \acrshort{BNF}, le script s'appuie en priorité sur cette source. L'identifiant de la langue est souvent disponible dans les données \Gls{unimarc}. L'identifiant \og{}fre\fg{} indique le français moderne. Même si la langue identifiée dans le \textit{manifest} \acrshort{IIIF} est le français, les données \Gls{unimarc} du catalogue de la \acrshort{BNF} pourrait en préciser la période, par exemple le moyen français (identifiant \og{}frm\fg{}). Même si notre modélisation du \texttt{<profileDesc>} est simple, il s'appuie sur deux sources de données et présente l'identifiant standardisé de la langue utilisé dans le document. Cela permet de filtrer les documents traités par notre pipeline selon la langue.

\begin{figure}[htp]
\centering
\begin{lstlisting}[language=XML]
<profileDesc>
	<langUsage>
		<language ident="fre">français</language>
	</langUsage>
</profileDesc>
\end{lstlisting}
\caption{La description non bibliographique de la source (\texttt{<profileDesc>})}
\label{fig:profileDesc}
\end{figure}


\section{La description technique (\texttt{<encodingDesc>})}
Le dernier composant du \texttt{<teiHeader>} est une description technique de l'encodage. Balisée dans l'élément \texttt{<encodingDesc>}, la description technique informe sur la manière dont l'encodage a été produit. Tout projet scientifique devrait pouvoir reproduire ses résultats. La ressource doit donc renseigner l'ensemble des étapes et des méthodes mises en place pour aboutir à la production du texte présent dans le fichier. L'élément \acrshort{TEI} \texttt{<appInfo>} renseigne sur le logiciel \acrshort{HTR} qui a permis de prédire l'analyse de la mise en page et le texte depuis les images numériques.

\begin{figure}[htp]
\centering
\begin{lstlisting}[language=XML]
<encodingDesc>
	<appInfo>
		<application ident="Kraken" version="3.0.13">
			<label>Kraken</label>
			<ptr target="https://github.com/mittagessen/kraken"/>
		</application>
	</appInfo>
	<classDecl>
		<taxonomy xml:id="SegmOnto">
			<bibl>
				<title>SegmOnto</title>
				<ptr target="https://github.com/segmonto"/>
			</bibl>
			<category xml:id="SegmOntoZones">
				<catDesc xml:id="MainZone">
					<title>MainZone</title>
					<ptr target="https://segmonto.github.io/gd/gdZ/MainZone"/>
				</catDesc>
				<catDesc xml:id="TitlePageZone">
					<title>TitlePageZone</title>
					<ptr target="https://segmonto.github.io/gd/gdZ/TitlePageZone"/>
				</catDesc>
<!-- more SegmOnto Zones --->
			</category>
			<category xml:id="SegmOntoLines">
				<catDesc xml:id="DefaultLine">
					<title>DefaultLine</title>
					<ptr target="https://segmonto.github.io/gd/gdL/DefaultLine"/>
				</catDesc>
				<catDesc xml:id="HeadingLine">
					<title>HeadingLine</title>
					<ptr target="https://segmonto.github.io/gd/gdL/HeadingLine"/>
				</catDesc>
<!-- more SegmOnto Lines --->
			</category>
		</taxonomy>
	</classDecl>
</encodingDesc>
\end{lstlisting}
\caption{La description non bibliographique de la source (\texttt{<profileDesc>})}
\label{fig:profileDesc}
\end{figure}


Le \texttt{<encodingDesc>} informe également sur la syntaxe qui a permis de structurer le texte encodé en utilisant l'élément \texttt{<classDecl>}. La \textit{déclaration des classes} (\texttt{<classDecl>}) porte sur les classes qui ont permis d'organiser le texte prédit. Dans notre modélsiation, les classes du vocabulaire \textit{SegmOnto} sont attribuées aux lignes de texte et aux zones qui contenaient le texte. Puisque les lignes de texte et les zones portent les noms du vocabulaire \textit{SegmOnto}, le \texttt{<classDecl>} fournit une description de chaque classe. La Figure~\ref{fig:profileDesc} montre comment la description de classe est balisée dans l'élément \texttt{<catDesc>} qui lui-même se trouve balisé dans la catégorie (\texttt{<category>}) de la classe, soit la ligne soit la zone.

\end{document}
\documentclass[../main.tex]{subfiles}