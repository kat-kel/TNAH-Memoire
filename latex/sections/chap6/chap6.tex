\documentclass[class=article, crop=false]{standalone}
\usepackage[subpreambles=true]{standalone}
\usepackage{fontspec}
\usepackage{hyperref}
\usepackage{import}
\usepackage{blindtext}

% !TEX TS-program = xelatex

%Example of code
\usepackage{listings}
\usepackage{color}
\definecolor{codegray}{rgb}{0.5,0.5,0.5}
\definecolor{codepurple}{rgb}{0.58,0,0.82}
\definecolor{cyan}{rgb}{0.0,0.6,0.6}
\definecolor{codegreen}{rgb}{0,0.6,0}
\definecolor{backcolour}{rgb}{0.95,0.95,0.92}

\lstdefinelanguage{XML}{
  backgroundcolor=\color{backcolour},  
  basicstyle=\ttfamily\footnotesize,
  morestring=[s]{"}{"},
  moredelim=[s][\color{black}]{>}{<},
  morecomment=[s]{!--}{--},
  commentstyle=\color{codegreen},
  moredelim=[s][\color{red}]{\ }{=},
  stringstyle=\color{blue},
  identifierstyle=\color{cyan},
  numberstyle=\tiny\color{codegray},
  breakatwhitespace=false,         
    breaklines=true,                 
    captionpos=b,                    
    keepspaces=true,                 
    numbers=left,                    
    numbersep=5pt,                  
    showspaces=false,                
    showstringspaces=false,
    showtabs=false,                  
    tabsize=2
}

%Code listing style named "json"
\lstdefinestyle{json}{
  backgroundcolor=\color{backcolour}, 
  basicstyle=\ttfamily\footnotesize,
  commentstyle=\color{codegreen},
  numberstyle=\tiny\color{codegray},
  basicstyle=\ttfamily\footnotesize,
  breakatwhitespace=false,         
  breaklines=true,                 
  captionpos=b,                    
  keepspaces=true,                 
  numbers=left,                    
  numbersep=5pt,                  
  showspaces=false,                
  showstringspaces=false,
  showtabs=false,                  
  tabsize=2
}

\begin{document}

\section{Uniquement l'essentiel}

\subsection{Documents de divers types et de plusieurs époques}

Expliquer le défi de modéliser un \texttt{<teiHeader>} qui est à la fois assez généralisé pour convenir aux divers documents et assez précisé pour servir aux utilisateurs et à la recherche.

\subsection{Exemples des métadonnées souhaitées}

Donner un exemple des métadonnées d'un imprimé (cf.\,fig.~\ref{fig:metadata-print}) :

\begin{figure}[ht]
\centering
\begin{lstlisting}[language=XML]
<sourceDesc>
	<biblStruct>
		<monogr>
			<author xml:id="author">
				<persName>
					<surname>Balzac</surname> <!-- auteur -->
					<forename>Honoré</forename>
				</persName>
			</author>
			<title>The Wild Ass's Skin</title> <!-- titre -->
			<editor role="translator">Ellen Marriage</editor>
			<editor role="preface">George Saintsbury</editor>
			<pubPlace key="FR">Paris</pubPlace>
			<imprint>
				<pubPlace>London</pubPlace> <!-- lieu de publication -->
				<publisher>Dent</publisher> <!-- éditeur -->
				<date when="1906">1906</date> <!-- date de publication -->
			</imprint>
		</mongr>
	</biblStruct>
\end{lstlisting}
\caption{Exemple des métadonnées d'un imprimé encodées en TEI}
\centering
(emprunté de \texttt{teibyexample.org} -- à changer)
\label{fig:metadata-print}
\end{figure}

Et donner un exemple d'un incunable (cf.\,fig.~\ref{fig:metadata-incunable})  :

\begin{figure}[ht]
\centering
\begin{lstlisting}[language=XML]
<sourceDesc>
	<msDesc>
		<msContents>
			<biblStruct>
				<monogr>
					<author xml:id="author">
						<persName>
							<surname>Tory</surname> <!-- auteur -->
							<forename>Geoffroy</forename>
						<persName>
					</author>
					<title>Champ fleury</title> <!-- titre -->
					<imprint>
						<pubPlace>Paris</pubPlace> <!-- lieu de publication / lieu d'apparition -->
						<publisher>
							<persName>
								<surname>Gourmont</surname> <!-- éditeur -->
								<forename>Gilles de</forename>
							</persName>
						</publisher>
					</imprint>
				</mongr>
			</biblStruct>			
\end{lstlisting}
\caption{Exemple des métadonnées d'un incunable encodées en TEI}
\centering
(emprunté du cours TEI de J-B Camps 2015 -- à changer)
\label{fig:metadata-incunable}
\end{figure}

Expliquer comment l'objectif du projet \textit{Gallic(orpor)a} de traiter des documents d'une diachronie longue pose un défi à la récupération et encodage généralisée des métadonnées.

\section{Où se trouvent les métadonnées des sources de Gallica}

Présenter les deux sources de métadonnées ciblées par l'application \texttt{alto2tei}.

\subsection{L'IIIF Image API}

L'API du manifest IIIF contient des données rudimentaires sur le document. Elles sont envoyées dans un format JSON. Donner un exemple (cf.\,fig.~\ref{fig:metadata-yaml}) :

\begin{figure}[ht]
\centering
\lstset{style=json}
\begin{lstlisting}[language=Python]
{"Metadata":
		{
		"Label": "Title",
		"Value": "The Wild Ass's Skin", # titre
		
		"Label": "Creator",
		"Value": "Honoré Balzac" # auteur
		}
}
\end{lstlisting}
\caption{Exemple des métadonnées envoyées par l'API IIIF}
\label{fig:metadata-yaml}
\end{figure}

\subsection{L'API SRU de la BnF}

L'API du catalogue général de la BnF contient des données bien précises sur le document. Elles sont envoyées dans un format XML-Unimarc. Donner un exemple (cf.\,fig.~\ref{fig:metadata-unimarc}).

\begin{figure}[ht]
\centering
\begin{lstlisting}[language=XML]
	<mxc:datafield tag="200" ind1="1" ind2=" ">
		<mxc:subfield code="a">The Wild Ass's Skin</mxc:subfield> <!-- titre -->
		<mxc:subfield code="b">Texte imprimé</mxc:subfield>
	</mxc:datafield>
	<mxc:datafield tag="210" ind1=" " ind2=" ">
		<mxc:subfield code="a">Londres</mxc:subfield> <!-- lieu de publication -->
		<mxc:subfield code="c">Dent</mxc:subfield> <!-- éditeur -->
		<mxc:subfield code="d">1906</mxc:subfield> <!-- date de publication -->
	</mxc:datafield>
[...]
	<mxc:datafield tag="700" ind1=" " ind2="|">
		<mxc:subfield code="a">Balzac</mxc:subfield> <!-- auteur -->
		<mxc:subfield code="b">Honoré</mxc:subfield>
	</mxc:datafield>
\end{lstlisting}
\caption{Exemple des métadonnées envoyées par l'API SRU de la BnF}
\label{fig:metadata-unimarc}
\end{figure}

\section{Une solution}

Présenter les métadonnées du document qu'on a déterminé d'être essentielle / assez généralisées parmi les divers types de document.

\end{document}
\documentclass[../main.tex]{subfiles}