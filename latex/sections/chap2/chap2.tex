% PREAMBULE
% !BIB TS-program = biber
% !TEX TS-program = xelatexmk
% ITEX TS-program = latex
% !TEX spellcheck = French

\documentclass[class=article, crop=false]{standalone}
\usepackage[subpreambles=true]{standalone}
\usepackage{import}
\usepackage{blindtext}
\usepackage{fontspec}
\usepackage[french]{babel}
\usepackage{csquotes}
\usepackage{url}

%%%%%%%%%%%%%%%%%%%%%%%%
%			REFERENCES
% le package hyperref avec des options, si en local
\usepackage[pdfusetitle, pdfsubject ={Mémoire TNAH}, pdfkeywords={les mots-clés}]{hyperref}
\usepackage[backend=biber, sorting=nyt, style=verbose-ibid]{biblatex}
\addbibresource{../../../bib.bib}


\begin{document}
	
\section{L'enjeu de l'encodage}
Un ordinateur fait ce que nous lui disons à faire. Si nous lui disons de rassembler toute note écrite dans les marges d'un livret autographe du librettiste Eugène Scribe, par exemple, l'ordinateur va chercher ce que nous lui avons dit se constitue une note en marge. Si nous lui avons dit qu'une note se constitue d'une suite de lignes de texte taguée par l'étiquette \textit{MarginalZone}, l'ordinateur va nous renvoie toutes les lignes de texte imbriquées dans une telle zone dans ses données d'entrée. Mais si les données qu'il traite n'appliquaient pas systématiquement l'étiquette \textit{MarginalZone} à toute note en marge, si elles appliquaient de temps en temps l'étiquette \textit{MarginalNoteZone}, par exemple, l'ordinateur ne parviendrait pas à récupérer tout ce que nous attendons. Les données d'entrée sont à faute, ainsi que celui qui les a créé. Si nous aillions produire une édition numérique du livret autographe à partir des données ainsi traitées, en faisant que toute note en marge soit indiquée, notre édition manquerait certaines notes et nous ne saurions pas lesquelles.

Le pipeline du projet \textit{Gallic(orpor)a} vise à produire une ressource numérique sur laquelle les chercheurs et les lecteurs peuvent compter. Il est important que la transcription produite par les modèles \acrshort{HTR} et transformée en le document \acrshort{TEI} soit aussi juste et complète que possible. Si les modèles qui ont produit la transcription représentée dans la ressource n'avaient pas appliqué systématiquement des étiquettes aux blocs de texte, les processus plus tard dans le pipeline, surtout mon application \texttt{alto2tei}, n'arriveraient pas à produire une ressource qui portent justement sur le document source.

L'équipe de \textit{Gallic(orpor)a} a choisi de mettre en pratique le vocabulaire codicologique du projet \textit{SegmOnto}. Son vocabulaire est l'une de plusieurs solutions proposées qui ont pour but de standardiser les étiquettes appliquées aux lignes de texte et aux régions d'une page. Le projet \textit{SegmOnto} reprend un défi qui a été franchi par plusieurs chercheurs depuis le vingtième siècle, comme j'explique prochainement. Par rapport aux autres projets, le vocabulaire de \textit{SegmOnto} convient le mieux aux enjeux du projet \textit{Gallic(orpor)a} parce qu'il a été développé pour un contexte informatique. L'équipe de \textit{Gallic(orpor)a} a entraîné les modèles du projet sur le vocabulaire afin qu'ils appliquent les étiquettes aux documents traités par son pipeline. En outre, la standardisation du vocabulaire \textit{SegmOnto} permet l'application \texttt{alto2tei} de filtrer les prédictions des modèles et leur appliquer les étiquettes du schéma \acrshort{TEI}.

\section{Les solutions proposées auparavant}

Plusieurs projets ont proposé des solutions pour la description normalisée des documents sources. Hors de la France, des ontologies ont été élaborées notamment en anglais et pour les manuscrits médiévaux. Un exemple important est la base de données \textit{DigiPal} (Digital Resource and Database of Palaeography, Manuscripts and Diplomatic), qui n'est plus mise à jour, mais qui a été développée au sein du département des humanités numériques à King's College London.\autocite{DigiPalDigitalResource2011} L'un de ses auteurs, Peter Stokes, travaille actuellement en France et continue dans la même veine en contribuant au projet \textit{SegmOnto} qui était développé dans un environnement francophone, même si son vocabulaire est rédigé en anglais\autocite{gabaySegmOnto2021}. Mais les projets \textit{DigiPal} et \textit{SegmOnto} ne sont pas les premiers projets à essayer d'élaborer un lexique pour les documents historiques. Avant leur création, la codicologie en France et en Europe suivait le modèle de Denis Muzerelle et son \textit{Vocabulaire codicologique}. Ce dernier n'était pas développé pour une utilisation informatique, mais il servait comme modèle pour les projets du même esprit codicologique.

\subsection{Le Vocabulaire international de la codicologie}

En 1985, Denis Muzerelle a conçu un vocabulaire codicologique qui avait pour but de fournir aux médiévistes  des termes communs afin de décrire les différents aspects d'un manuscrit\autocite{muzerelleVocabulaireCodicologiqueRepertoire1985}. Après sa publication, l'ouvrage de D.~Muzerelle a été traduit en d'autres langues. Marilena Maniaci a publié une version italienne du \textit{Vocabulaire codicologique} en 1996\autocite{maniaciTerminologiaLibroManoscritto1996}. Pilar Ostos, Luisa Pardo, et Elena RodrÃguez en ont créé un pour l'espagnol l'année suivante\autocite{ostosVocabularioCodicologIaVersion1997}. Parfois appelée le \textit{Vocabulaire international de la codicologie}, l'édition multilingue du \textit{Vocabulaire codicologique} initiée par D. Muzerelle en 1985 n'est plus mis à jour.

\subsection{La Codicologia}

Aujourd'hui, la paléographie et l'étude des manuscrits peuvent s'appuyer sur l'application web \textit{Codicologia} qui réunit le \textit{Vocabulaire codicologique} initiée par D. Muzerelle ainsi que deux autres bases de données similaires: le projet multilingue \textit{Lexicon} et le \textit{Glossaire codicologique arabe}. Bien qu'ils soient spécialisées dans d'écritures différentes, les projets s'appuient tous les trois sur la description des manuscrits. Le \textit{Vocabulaire codicologique} a été développé pour les manuscrits en écriture latine. Piloté par Philippe Bobichon, le projet \textit{Lexicon} présente un vocabulaire en français pour décrire les manuscrits écrits en latin, roman, grec, hébreu, et arabe\footcite{bobichonLexiconMisePage2009}. Un vocabulaire spécialisé plus profondément pour l'arabe a été élaboré dans le \textit{Glossaire codicologique arabe} d'Anne-Marie Eddé et Marc Geoffroy\footcite{GlossaireCodicologiqueFrancaisarabe2002}. Ce dernier a été conçu au sein de l'Institut de recherche et d'histoire des textes (IRHT) d'après les modèles de Muzerelle et le vocabulaire codicologique en arabe d'Adam Gacek\footcite{gacekArabicManuscriptTradition2001}.

L'application web \textit{Codicologia} rassemble ces projets et présente un vocabulaire bien étendu. Par exemple, \textit{Codicologia} fournit 15 termes pour décrire une faute d'écriture dans un manuscrit. Certains de ses termes possèdent eux-mêmes plusieurs définitions fournies par les diverses bases de données. Le terme \textit{caviarder}, par exemple, a une définition courte dans le vocabulaire français de Muzerelle. 

\begin{displayquote}
Supprimer un mot, un passage..., en le recouvrant largement d'encre, de façon à ce qu'il ne puisse être lu\footcite{muzerelleCaviarder2011}.
\end{displayquote}

\noindent Selon le \textit{Lexicon} de Bibichon, par contre, le terme \textit{caviarder} est défini de manière plus détaillée et vise à expliquer l'étymologie du mot afin de préciser son usage en contexte.

\begin{displayquote}
Le mot [\textit{caviarder}] apparaît en 1907 (noircir à l'encre) : il désigne alors un procédé appliqué par la censure russe, sous Nicolas I\textsuperscript{er}. Dans certains manuscrits grecs, le détail rempli d'encre est surmonté d'un point et d'un trait court destinés à le neutraliser. Ce procédé est très souvent utilisé parmi d'autres, pour la censure des manuscrits hébreux effectuée sous l'autorité de l'inquisition, en Italie, à la fin du \textsc{xvi}\textsuperscript{e} siècle et au début du \textsc{xvii}\textsuperscript{e}\autocite{bobichonCaviarder2011}.
%Attention, il y a des problèmes dans la forme de la référence en note de bas de page.


\end{displayquote}

\noindent Étant élaboré à partir d'un corpus très diversifié, le \textit{Lexicon} de P. Bibichon a moins de termes que le \textit{Vocabulaire codicologique}, mais ses termes sont plus généralistes. Le vocabulaire de Muzerelle, par contre, fait plus de distinctions entre les différents aspects d'un manuscrit et donc a plus de termes distincts par rapport aux deux autres vocabulaires de l'application \textit{Codicologia}.

En réunissant les trois bases de données, sans en privilégier aucune, le \textit{Codicologia} de l'IRHT présente un vocabulaire codicologique vraiment vaste. 
L'application \textit{Codicologia} cherche à proposer à la communauté scientifique des termes communs pour décrire les manuscrits, malgré son étendue.
Ayant plus de deux milles termes en français---certains d'entre eux ont eux-mêmes plusieurs définitions---la solution proposée par \textit{Codicologia} livre un vocabulaire documenté, mais trop étendu pour être appliqué dans une approche informatique.

Sans un corpus d'entraînement gigantesque, qui couterait une somme énorme, l'apprentissage automatique ne peut pas faire de distinction au niveau des termes conçus par Muzerelle et les autres auteurs des bases de données de \textit{Codicologia}. Aujourd'hui, un modèle ne peut pas apprendre des milliers d'étiquettes différents et distinguer, par exemple, 15 types de faute d'écriture. Un humain peut le faire, et pour cette raison les bases de données de \textit{Codicologia} sont utiles. Mais leurs vocabulaires ne conviennent pas bien à une approche informatique.  

\section{Les \textit{guidelines} de \textit{SegmOnto}}

Le projet \textit{SegmOnto} propose un vocabulaire plus petit qui peut pourtant décrire une grande diversité de documents historiques, soit des manuscrits, soit des imprimés. Cet objectif est encore plus compliqué à achever que le vocabulaire spécialisé aux manuscrits qu'ont réalisé les projets \textit{Codicologia} et \textit{DigiPals}.
Décrire les documents en diachronie longue, et sans préférence d'une écriture en particulier, exige un équilibre délicat entre la généralité et la particularité. Pour y arriver, les \textit{guidelines} du projet \textit{SegmOnto} limitent le nombre de termes de son vocabulaire, sans perdre de distinction importante entre eux.

Les \textit{guidelines} se divisent en deux catégories~: les \og{}zones\fg{} et les \og{}lines\fg{}. La première désigne des régions sur la page qu'elles ont du texte ou non, telle comme la décoration ou l'enluminure. La plupart de temps, l'étiquette de la ligne veut décrire les différents types de lignes de texte. Cependant, dans le vocabulaire \textit{SegmOnto}, une ligne peut aussi tracer la ligne d'une partition musicale ou être une ligne réelle sur la page qui n'oriente pas d’autres systèmes d'écriture, telle qu'une ligne qui divise la page en deux. Chacune des deux catégories, les lignes et les zones, se compose d'une liste des étiquettes. Et chacune d'étiquettes cherche à parvenir à l'équilibre entre la généralité et la particularité. Une étiquette devrait pouvoir être appliquée soit à un manuscrit, soit à un imprimé, quels que soient la langue ou le type d'écriture.

\subsection{Les zones}
\begin{itemize}
\item \textbf{CustomZone}~: une zone qui ne convient pas à aucune d'autres catégories de zone.
\item \textbf{DamageZone}~: une zone qui contient des marques de dégâts sur le document source, tel qu'un trou.
\item \textbf{DecorationZone}~: une zone qui contient un élément graphique, y compris de la peinture et les petits dessins dans la marge de la page.
\item \textbf{DigitizationArtefactZone}~: une zone qui contient un item qui n'appartient pas au document source, mais est lié au processus de la numérisation, tel qu'une règle pour montrer la mesure du document. 
\item \textbf{DropCapitalZone}~: une zone qui contient une initiale ; l'initiale peut prendre l'espace de plusieurs lignes de texte ou porter une décoration importante, tel que de l'historicisation, l'ornementation, ou des dessins.
\item \textbf{MainZone}~: une zone qui contient le texte principal du document source.
\item \textbf{MarginTextZone}~: une zone qui contient le texte dans la marge du document source.
\item \textbf{MusicZone}~: une zone qui contient une partition musicale.
\item \textbf{NumberingZone}~: une zone qui contient des numéros de page, y compris les numéros rédigés en chiffres romans.
\item \textbf{QuireMarksZone}~: une zone qui contient des notes en bas page destinées à la fabrication du document source pour garder les pages dans le bon ordre.
\item \textbf{RunningTitleZone}~: une zone qui contient une version du titre du document ou d'une section du document qui se trouve en tête de la page.
\item \textbf{SealZone}~: une zone qui contient un sceau sur le document source.
\item \textbf{StampZone}~: une zone qui contient l'empreinte d'un tampon sur le document source.
\item \textbf{TableZone}~: une zone qui contient une table. 
\item \textbf{TitlePageZone}~: une zone souvent sur l'une des premières pages du document source qui contient toutes les informations concernant le titre et l'édition du document.
\end{itemize}

\subsection{Les lignes}
\begin{itemize}
\item \textbf{CustomLine}~: une ligne qui ne convient pas à aucune d'autres catégories de ligne.
\item \textbf{DefaultLine}~: une ligne qui contient du texte attendu dans la zone.
\item \textbf{DropCapitalLine}~: une ligne qui contient l'initiale. 
\item \textbf{HeadingLine}~: une ligne qui contient le texte d'un titre, tel que celui d'une section ou d'un chapitre.
\item \textbf{InterlinearLine}~: une ligne qui traverse la page pour marquer une limite.
\item \textbf{MusicLine}~: une ligne de la portée d'une partition.
\end{itemize}
	
\end{document}
\documentclass[../main.tex]{subfiles}