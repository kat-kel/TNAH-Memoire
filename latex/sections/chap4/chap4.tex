\documentclass[class=article, crop=false]{standalone}
\usepackage[subpreambles=true]{standalone}
\usepackage{import}
\usepackage{blindtext}
\begin{document}


\section{La reconnaissance du texte et des segments}

Le premier étape du pipeline \textit{Gallic(orpor)a} est la reconnaissance du texte sur les images numérisées. Pour arriver d'un rassemblement de pixels au caractère d'un système d'écriture, il faut un modèle HTR qui sait chercher dans les pixels les configurations des caractères. Avec la reconnaissance de texte, il faut un deuxième modèle qui sait reconnaître les régions cohérentes sur la page. Ce dernier modèle cherchent aussi dans les pixels pour les configurations consistantes, mais au lieu de reconnaître dedans des caractères, il relève les polygons ou les rectangles qui contient une entité cohérente.

\subsection{La création des données d'entraînement}

Le pipeline a besoin d'une structure de fichier rigide qu'il construit lors de la récupération des images depuis l'API de la \acrshort{BNF}. Sinon, l'utilisatrice ou l'utilisateur doit l'imiter en appuyant sur ses propres images numériques d'un document source. Mais les images doivent être disponible en ligne par l'API de \textit{l'\Gls{iiif}} (\acrshort{IIIF}) pour que les métadonnées de la source de l'image soient encodées et accessibles par la requête.

\begin{center}
\begin{verbatim}
data/
|____ARK1/
|        |____image1.jpg
|        |____image2.jpg
|        |____image3.jpg
...
|____ARK2/
|        |____image1.jpg
|        |____image2.jpg
|        |____image3.jpg
...
\end{verbatim}
\end{center}

\noindent Toute image numérique doit se trouve dans un dossier qui porte comme nom l'identifiant ARK (\textit{Archival Resource Key}) du document source. Cette clef est importante pour la diffusion et le requêtage des images numériques par les APIs \acrshort{IIIF}. Le pipeline a aussi besoin de cette clef ainsi que de l'association évidente entre elle et les images qui appartiennent au document, qui se donne grâce à la structure de fichier.

Avec cet identifiant ARK, donné par la structure de fichier, le pipeline récupère les métadonnées du document source. Il recherche les informations sur le document source en passant une requête à l'API de l'image \acrshort{IIIF}~; cet API est maintenu par l'institution patrimonial qui se charge de la conservation et/ou de la diffusion en ligne du document. Les métadonnées récupérés de l'API IIIF, géré par l'institution hôte du document numérique, sont liées---si possible---avec des autres sources de données. Dans l'exemple des documents numériques de la base de données Gallica, le pipeline récupère les métadonnées quant au document source depuis l'API \acrshort{IIIF} que la \acrshort{BNF} met à disposition. Si cette requête a réussi, le pipeline va récupérer les données du catalogue général de la \acrshort{BNF} en passant une requête à l'API \textit{SRU} de la \acrshort{BNF} qui veut dire, en anglais, le \textit{Search/Retrieve via URL}. Pour terminer, le pipeline recherche encore des métadonnées dans le catalogue du Système Universitaire de Documentation (SUDOC).

\subsection{L'entraînement des modèles}

\subsection{Les modèles prédisent le texte}

\section{La reconstitution des données}

\section{L'analyse linguistique}

\subsection{La création des données d'entraînement}

\subsection{L'entraînement des modèles}

\subsection{Les modèles analysent le texte prédit}

\section{Le texte pré-édité}


\end{document}
\documentclass[../main.tex]{subfiles}