\documentclass[class=article, crop=false]{standalone}
\usepackage[subpreambles=true]{standalone}
\usepackage{import}
\usepackage{blindtext}
\begin{document}

\section{Le contexte du projet}

Présenter les institutions qui soutiennent le projet (École nationale des chartes, Inria, Université de Genève, DataLab de la BnF) et les projets qui l'ont précédé et sur lesquels \textit{Gallic(orpor)a} compte.

\section{L'objectif du projet}

Le projet a pour but d'arriver des images numérisées à un document numérique sans besoin d'un éditeur. L'idée est qu'avec le produit du pipeline, qui sera un texte prédit et pré-éditorialisé, un individu qui n'est pas forcement spécialisé dans l'informatique peut facilement le prendre et effectuer des analyses et/ou éditer le texte du document dont les images de ses pages ont été traités.

\section{Le pipeline}

Présenter le pipeline. Réutiliser l'information qu'on a générée pour le présenter à la BnF.

\end{document}
\documentclass[../main.tex]{subfiles}