\documentclass[class=article, crop=false]{standalone}
\usepackage[subpreambles=true]{standalone}
\usepackage{import}
\usepackage{blindtext}
\begin{document}

\section{Portée des données d'entraînement}
\label{data:portee}
\begin{center}
\begin{tabular}[c]{| *{6}{| m{1.6cm}} | m{3cm} ||}
\hline
Type & Genre & Forme & Écriture & Siècle & Langue & Titre \\
\hline \hline
%2
manuscrit & poésie & vers & gothique & 13 & fro & Français 20050   - chansonnier de Saint-Germain-des-Près \\
\hline
%3
manuscrit & récit & prose & gothique & 13 & fro & Français 23117, légendier\\
\hline
%4
manuscrit & récit & prose & gothique & 13 & fro & Français 6447, légendier\\
\hline
%5
manuscrit & récit & prose & gothique & 13 & fro & NAF 23686, légendier\\
\hline
%6
manuscrit & récit & prose & gothique & 13 & fro & Français 13496, légendier\\
\hline
%7
manuscrit & récit & vers & gothique & 13 & fro & Français 860 - Roland, Gaydon, Ami et Amile, Jourdain de Blaye, Auberi le Bourguignon\\
\hline
%8
manuscrit & récit & vers & gothique & 13 & fro & Français 12615  - chansonnier de Noailles\\
\hline
%9
manuscrit & récit & vers & gothique & 13 & fro & Français 1443 - Garin le Loherain (C) et Girbert de Metz\\
\hline
%10
manuscrit & récit & vers & gothique & 13 & fro & Français 12603 - Fierabras, mais aussi Chevalier as deux espees, Chevalier au Lyon, Eneas, Wace:Brut, Adenet:EnfancesOgier, Fabliaux, Lai de l'Ombre, Raoul de Houdenc:Songe d'Enfer, Fables de Marie de Fance\\
\hline
%11
manuscrit & récit & vers & gothique & 13 & fro & Français 12558  - Chevalier au Cygne, Chanson d'Antioche, Chanson de Jérusalem\\
\hline
%12
manuscrit & récit & vers & gothique & 13 & fro & Français 1598 - Aspremont, Anséïs de Carthage\\
\hline
%13
manuscrit & récit & prose & gothique & 13 & fro & Français 185, légendier\\
\hline
\end{tabular}
\end{center}


\end{document}
\documentclass[../main.tex]{subfiles}