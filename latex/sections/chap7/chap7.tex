% PREAMBULE

% !BIB TS-program = biber
% !TEX TS-program = xelatexmk
% ITEX TS-program = latex

% !TEX spellcheck = French

\documentclass[class=article, crop=false]{standalone}
\usepackage[subpreambles=true]{standalone}
\usepackage{import}
\usepackage{blindtext}
\usepackage{fontspec}
\usepackage[french]{babel}
\usepackage{caption}
\usepackage{subcaption}
\usepackage{csquotes}
\usepackage{url}

%%%%%%%%%%%%%%%%%%%%%%%%
%			REFERENCES
% le package hyperref avec des options, si en local
\usepackage[pdfusetitle, pdfsubject ={Mémoire TNAH}, pdfkeywords={les mots-clés}]{hyperref}
\usepackage[backend=bibtex, sorting=nyt, style=verbose-ibid]{biblatex}
\addbibresource{../../../bib.bib}

%%%%%%%%%%%%%%%%%%%%%%%%
%			GLOSSAIRE
\usepackage[acronym]{glossaries}
\newglossaryentry{htr}
{
    name=Handwritten Text Recognition,
    description={La reconnaissance du texte écrit sur une image numérique}
}
\newacronym{HTR}{HTR}{Handwritten Text Recognition}

\newglossaryentry{ocr}
{
    name=Optical Character Recognition,
    description={La reconnaissance des polices du texte sur une image numérique}
}
\newacronym{OCR}{OCR}{Optical Character Recognition}

\newglossaryentry{Inria}
{
    name=Inria,
    description={Institut national de recherche en sciences et technologies du numérique}
}
\newacronym{INRIA}{Inria}{Institut national de recherche en sciences et technologies du numérique}

\newacronym{almanach}{ALMAnaCH}{Automatic Language Modelling and Analysis \& Computational Humanities}

\newglossaryentry{enc}
{
    name=École nationale des chartes,
    description={Grande école bla bla bla}
}
\newacronym{ENC}{ENC}{École nationale des chartes}

\newglossaryentry{HTR-United}
{
    name=HTR-United,
    description={HTR-United is a catalog and an ecosystem for sharing and finding ground truth for optical character or handwritten text recognition (OCR/HTR)}
}

\newglossaryentry{CLab}
{
	name=CREMMALab,
	description={Consortium pour la reconnaissance
d’'écriture manuscrite des matériaux anciens}
}
\newacronym{CREMMA}{CREMMA}{Consortium Reconnaissance
d’Écriture Manuscrite des Matériaux Anciens}

\newglossaryentry{tei}
{
	name={Text Encoding Initiative},
	description={Normes internationales de l'encodage des documents texts}
}
\newacronym{TEI}{TEI}{Text Encoding Initiative}

\newglossaryentry{iiif}
{
	name={International Image Interoperability Framework},
	description={Normes internationales de l'exploitation des images numériques et de leurs métadonnées par API}
}
\newacronym{IIIF}{IIIF}{International Image Interoperability Framework}

\newacronym{ALTO}{ALTO}{Analyzed Layout and Text Object}

\newacronym{XML}{XML}{eXtensible Markup Language}

\newacronym{BNF}{BnF}{Bibliothèque nationale de France}

\newacronym{RDF}{RDF}{Resource Description Framework}

\newacronym{TAL}{TAL}{Traitement automatique des langues}

\newacronym{ARK}{ARK}{Archival Resource Key}

\newacronym{DTS}{DTS}{Distributed Text Services}

\newglossaryentry{iiifapi}
{
	name={IIIF Image API},
	description={Un service de web qui renvoie une image suite à une requête standardisée HTTP(S). L'URI peut préciser la région, la taille, la rotation, la qualité, les caractéristiques, et le format de l'image demandée.}
}
\newacronym{API}{API}{Application Programming Interface}

\newglossaryentry{odd}
{
	name={One Document Does it all},
	description={Un fichier XML TEI qui précise les règles d'un schéma TEI personnalisé.}
}
\newacronym{ODD}{ODD}{One Document Does it all}

\newacronym{JSON}{JSON}{JavaScript Object Notation}

\newacronym{HTML}{HTML}{HyperText Markup Language}

\newacronym{METS}{METS}{Metadata Encoding and Transmission Standard}

\newacronym{YAML}{YAML}{Yet Another Markup Language}

\newacronym{SRU}{SRU}{Search/Retrieve via URL}

\newglossaryentry{unimarc}
{
	name={UNIMARC},
	description={Une référence pour l’échange de données en format XML}
}

\newacronym{SUDOC}{SUDOC}{Système Universitaire de Documentation}

%%%%%%%%%%%%%%%%%%%%%%%%
%			DIAGRAM
\usepackage{tikz}
\usetikzlibrary{positioning}
\usetikzlibrary{calc, matrix, shapes.geometric, arrows}
\usepackage{pgfplots}
\usepackage{array}
\usepackage{tabularx}
\usepackage{graphicx}


%%%%%%%%%%%%%%%%%%%%%%%%
%			CODE
\usepackage{listings}
\usepackage{color}
\definecolor{codegray}{rgb}{0.5,0.5,0.5}
\definecolor{codepurple}{rgb}{0.58,0,0.82}
\definecolor{cyan}{rgb}{0.0,0.6,0.6}
\definecolor{codegreen}{rgb}{0,0.6,0}
\definecolor{backcolour}{rgb}{0.95,0.95,0.92}

\lstdefinelanguage{XML}{
  backgroundcolor=\color{backcolour},  
  basicstyle=\ttfamily\footnotesize,
  morestring=[s]{"}{"},
  moredelim=[s][\color{black}]{>}{<},
  morecomment=[s]{!--}{--},
  commentstyle=\color{codegreen},
  moredelim=[s][\color{red}]{\ }{=},
  stringstyle=\color{blue},
  identifierstyle=\color{cyan},
  numberstyle=\tiny\color{codegray},
  breakatwhitespace=false,         
    breaklines=true,                 
    captionpos=b,                    
    keepspaces=true,                 
    numbers=left,                    
    numbersep=5pt,                  
    showspaces=false,                
    showstringspaces=false,
    showtabs=false,                  
    tabsize=2
}

%Code listing style named "json"
\lstdefinestyle{json}{
  backgroundcolor=\color{backcolour}, 
  basicstyle=\ttfamily\footnotesize,
  commentstyle=\color{codegreen},
  numberstyle=\tiny\color{codegray},
  basicstyle=\ttfamily\footnotesize,
  breakatwhitespace=false,         
  breaklines=true,                 
  captionpos=b,                    
  keepspaces=true,                 
  numbers=left,                    
  numbersep=5pt,                  
  showspaces=false,                
  showstringspaces=false,
  showtabs=false,                  
  tabsize=2
}

%Code listing style named "python"
\definecolor{codepurple}{rgb}{0.58,0,0.82}

\lstdefinestyle{python}{
    backgroundcolor=\color{backcolour},   
    commentstyle=\color{codegreen},
    keywordstyle=\color{magenta},
    numberstyle=\tiny\color{codegray},
    stringstyle=\color{codepurple},
    basicstyle=\ttfamily\footnotesize,
    breakatwhitespace=false,         
    breaklines=true,                 
    captionpos=b,                    
    keepspaces=true,                 
    numbers=left,                    
    numbersep=5pt,                  
    showspaces=false,                
    showstringspaces=false,
    showtabs=false,                  
    tabsize=2
}

%%%%%%%%%%%%%%%%%%%%%%%%
%%%%%%%%%%%%%%%%%%%%%%%%
%			DOCUMENT
%%%%%%%%%%%%%%%%%%%%%%%%
%%%%%%%%%%%%%%%%%%%%%%%%
\begin{document}
J'ai créé l'application \texttt{alto2tei} afin que le pipeline \textit{Gallic(orpor)a} puisse aller des fichiers sortis des modèles \acrshort{HTR} vers une version préliminaire de la ressource numérique qu'il produit\footnote{Pour faire un petit résumé de ce que nous avons déjà abordé, la première étape du pipeline est la récupération des pages numérisées du document source. Ensuite le pipeline prédit le texte et la mise en page du fac-similé numérique en traitant chaque page avec des modèles \acrshort{HTR}. Les données produites par les modèles sont en format \acrshort{XML} \acrshort{ALTO}. Mais le pipeline veut présenter les données enrichies par les métadonnées et par l'analyse linguistique en format \acrshort{XML} \acrshort{TEI} puisque le schéma \acrshort{TEI} est plus utilisé et il convient mieux à l'édition du texte que le schéma \acrshort{ALTO}. Mon application \texttt{alto2tei} a complété le pipeline en construisant un document \acrshort{TEI} à partir des données des modèles \acrshort{HTR}.}. L'objectif est donc de transformer des données du format \acrshort{XML} \acrshort{ALTO} en un format \acrshort{XML} \acrshort{TEI}. Cependant, la création d'un document \acrshort{TEI} préliminaire n'est pas si simple que la transposition des données d'une structure dans une autre. Le schéma \acrshort{TEI} exige plus que cela. Dans le chapitre \ref{chap:metadata}, l'exposition de notre modélisation des métadonnées explique que la transcription d'un document n'est pas tout ce qu'il faut dans un document \acrshort{TEI}. Il faut aussi des métadonnées, mais elles ne se trouvent pas dans la transcription prédite par des modèles \acrshort{HTR}. L'application \texttt{alto2tei} a donc besoin de récupérer les métadonnées à partir des sources externes, qui contiennent des données autre que celles créées par le pipeline. 

Quelles métadonnées faut-il chercher~? Et où~? Dans le chapitre~\ref{chap:metadata}, je parle de trois objets de texte conceptuels pertinents dans notre modélisation des métadonnées~:  (1) la ressource numérique elle-même en format \acrshort{XML} \acrshort{TEI}, (2) le fac-similé numérique disponible depuis l'\acrshort{API} de la \acrshort{BNF} dont les images sont traitées par les modèles \acrshort{HTR}, et (3) le document physique qui est conservé par la \acrshort{BNF} et qui a été numérisé. Chacun est distinct et ses métadonnées se trouvent depuis des sources différentes. Afin d'informer les métadonnées de la ressource numérique, il faut donc s'appuyer sur les métadonnées de ces dits objets de texte.

Le pipeline du projet \textit{Gallic(orpor)a} a besoin de diversifier sa manière de récupérer des métadonnées. L'application \texttt{alto2tei} profite de quatre sources externes de données afin de renseigner sur les trois documents conceptuels concernés.
\begin{enumerate}
\item les métadonnées sur la ressource lexicographique numérique
\begin{itemize}
\item source~: fichier de configuration personnalisable
\item format~: \acrshort{YAML} (\textit{\acrlong{YAML}})
\item portée~: informations administratives portant sur la création de la ressource, y compris les droits d'utilisation et les autorités qui s'en chargent
\end{itemize}
\item les métadonnées sur le fac-similé numérique
\begin{itemize}
\item source~: manifest \acrshort{IIIF} renvoyé de l'\acrshort{IIIF} \acrshort{API}
\item format~: \acrshort{JSON} (\textit{\acrlong{JSON}})
\item portée~: données bibliographiques qui servent à identifier les exemplaires numériques traités par des modèles \acrshort{HTR}
\end{itemize}
\item les métadonnées sur le document source physique
\begin{itemize}
\item source~: réponse à la requête envoyée à l'\acrshort{API} \acrshort{SRU} (\textit{\acrlong{SRU}}) de la \acrshort{BNF}, qui interroge le catalogue général de la \acrshort{BNF}
\item format~: \Gls{unimarc} \acrshort{XML}
\item portée~: données bibliographiques, y compris la responsabilité du document source, sa création, et sa conservation actuelle
\end{itemize}
\end{enumerate}
Les métadonnées sur le document source physique s'appuient encore sur une autre source de données afin de récupérer où se trouve l'institution qui héberge le document. Dans le cadre du projet \textit{Gallic(orpor)a}, la réponse à cette question est toujours Paris, puisque la \acrshort{BNF} se situe au capital. Mais, afin d'éviter l'encodage dur, l'application \texttt{alto2tei} recherche cette donnée dans la base de données du \acrlong{SUDOC} (\acrshort{SUDOC}) puisque la localisation de l'institution hôte du document physique n'est pas encodée dans les données \Gls{unimarc} selon l'usage de la \acrshort{BNF}.

\section{La récupération des métadonnées}
Comme est expliqué dans l'exposition du pipeline entier (chap.~\ref{chap:pipeline}), l'application \texttt{alto2tei} s'appuie sur un système de fichier fixé. Cela lui permet d'accéder à l'identifiant \acrshort{ARK} (\acrlong{ARK}) du document dont les pages numérisées ont été traitées par les modèles \acrshort{HTR}. Cet identifiant donne la clef aux données sur le fac-similé numérique encodées dans le \textit{manifest} \acrshort{IIIF} (\acrlong{IIIF}). L'une des données dans le \textit{manifest} \acrshort{IIIF}, au moins pour les documents de la base de données Gallica, porte sur la relation entre le fac-similé numérique est la source physique à partir duquel le fac-similé numérique a été produit, voir la Figure~\ref{fig:relation}. Cette relation permet de rechercher les métadonnées du document physique dans le catalogue général de la \acrshort{BNF}.

\begin{figure}[ht]
\centering
\lstset{style=json}
\begin{lstlisting}[language=Python]
"metadata": [
	{
	"label": "Relation",
	"value": "Notice du catalogue : http://catalogue.bnf.fr/ark:/12148/cb30369299r"
	},
]
\end{lstlisting}
\caption{Une partie du \textit{manifest} \acrshort{IIIF} pour le fac-similé numérique d'\acrshort{ARK} \texttt{bpt6k1281160s}, un imprimé numérisé sur Gallica}
\label{fig:relation}
\end{figure}

Si la relation entre le document numérique sur Gallica et le document physique dans l'un des magasin de la \acrshort{BNF} n'est pas établie, l'application \texttt{alto2tei} n'abandonne pas tout essai de produire un document \acrshort{TEI}. Au lieu de s'arrêter, elle compte uniquement sur les données du \textit{manifest} \acrshort{IIIF}. Ces données sont vérifiées puisqu'elles portent sur le fac-similé numérique dont les images les modèles \acrshort{HTR} du pipeline ont traitées. Le fac-similé numérique est toujours l'un des objets de texte pertinents aux métadonnées de la ressource. L'application \texttt{alto2tei} peut donc créer deux genres de \texttt{<teiHeader>}, l'un avec un maximum d'information et l'autre, au cas où la relation entre le fac-similé numérique et le document physique n'était pas établie, avec un peu moins d'information.

\section{Les sources des métadonnées}
Six source de données servent à informer des éléments du \texttt{<teiHeader>}, mais l'une d'eux est un calcul simple de données internes au pipeline et l'autre est une donnée donnée au démarrage de l'application ou du pipeline. Elles sont (1)~le \textit{manifest} \acrshort{IIIF}, (2)~les données \Gls{unimarc} du catalogue général de la \acrshort{BNF}, (3)~la base de données du site web du \acrlong{SUDOC} (\acrshort{SUDOC}), (4)~le fichier de configuration, (5)~les fichiers \acrshort{ALTO} produits par des modèles \acrshort{HTR}, et (6)~la version de \textit{Kraken} utilisée par les modèles \acrshort{HTR}. Les trois premières sources exigent une requête à une \acrshort{API}. Le \textit{manifest} \acrshort{IIIF} est renvoyé par une requête à l'\acrshort{API} \acrshort{IIIF} du fac-similé numérique sur Gallica, selon les normes imposées par la \acrshort{BNF}. Les données \Gls{unimarc} du catalogue général de la \acrshort{BNF} sont accessibles à partir d'une requête à l'\acrshort{API} \acrshort{SRU} (\textit{\acrlong{SRU}}) de la \acrshort{BNF}. Enfin, les données du \acrlong{SUDOC} sont renvoyées à partir d'une requête à l'\textit{endpoint} \acrshort{API} du site \acrshort{SUDOC}. L'application \texttt{alto2tei} dispose directement du fichier de configuration ainsi que des fichiers \acrshort{ALTO} qu'elle traite.

\subsection{Le format du \textit{manifest} \acrshort{IIIF}}
\label{sec:manifest}
Le \textit{manifest} \acrshort{IIIF} est un format de données standardisé qui est envoyé par tout \acrshort{API} \acrshort{IIIF} selon la syntaxe de requête demandée par l'organisation qui gère l'\acrshort{API}. La \acrlong{BNF} exige la syntaxe suivante pour toute requête envoyée à l'\acrshort{API} \acrshort{IIIF} des fac-similés numériques mis sur Gallica~:
\begin{verbatim}
https://gallica.bnf.fr/iiif/ark:/12148/<ARK>/manifest.json
\end{verbatim}
Pour utiliser cette requête, il faut simplement remplacer la partie \texttt{<ARK>} par l'\acrshort{ARK} du document cherché. Grâce au système de fichier du pipeline \textit{Gallic(orpor)a}, l'\acrshort{ARK} du document est facilement disponible. Il est le nom du dossier qui contient les fichiers \acrshort{XML} \acrshort{ALTO} traités.

L'\acrshort{API} \acrshort{IIIF} renvoie les données dans un fichier en format JSON. La structure des données dedans peut varier un peu, mais certains composants sont toujours attendus. Le Tableau~\ref{tab:manifestExample} montre toute clef principale du \textit{manfiest} pour un fac-similé numérique mis en ligne sur Gallica par la \acrshort{BNF}. L'exemple prend le livret imprimé du ballet \textit{Les divers entretiens de la fontaine de Vaucluse} de 1649, dont le fac-similé a l'identifiant \texttt{bpt6k1513919s}. Les données les plus importantes sont celle imbriquées dans la clef \textit{metadata} et celles de la clef \textit{sequences}. La première contient une suite des pairs clef-valeur, chacun porte sur une métadonnée du fac-similé numérique. La dernière s'appuie sur les images numériques.

\begin{figure}[htp]
\begin{tabularx}{\textwidth}{ 
  | m{5em}
  | >{\arraybackslash}X 
  |}
\hline
\textbf{clef} & \textbf{valeur}\\
\hline \hline
@id & https://gallica.bnf.fr/iiif/ark:/12148/bpt6k1513919s/manifest.json\\
\hline
label & BnF, département Réserve des livres rares, RES-YF-1990\\
\hline
attribution & Bibliothèque nationale de France\\
\hline
license & https://gallica.bnf.fr/html/und/conditions-dutilisation-des-contenus-de-gallica\\
\hline
logo & https://gallica.bnf.fr/mbImage/logos/logo-bnf.png\\
\hline
related & https://gallica.bnf.fr/ark:/12148/bpt6k1513919s\\
\hline
description & Les divers entretiens de la fontaine de Vaucluse : ballet dansé à la grande salle du Roure l'année 1649...\\
\hline
metadata & \textit{suite des pairs qui portent sur les métadonnées du fac-similé numérique}\\
\hline
sequences & \textit{imbrication complexe pour contenir des données sur les images, dit canvases, du fac-similé numérique}\\
\hline
thumbnail & \textit{pair qui contient un lien pour l'aperçu du document}\\
\hline
context & http://iiif.io/api/presentation/2/context.json\\
\hline
\end{tabularx}

\caption{Les clefs principales du \textit{manifest} \acrshort{IIIF} d'un document sur Gallica (\acrshort{ARK} \texttt{bpt6k1513919s})}
\label{tab:manifestExample}
\end{figure}

Les données imbriquées dans la clef \textit{metadata} peuvent varier de document à document, même quand elles sont encodées par la même institution telle que la \acrlong{BNF}. Comme est expliqué dans le chapitre~\ref{chap:metadata}, il faut déterminer l'essentiel quant aux métadonnées à mettre dans le schéma \acrshort{TEI}. Il faut chercher les métadonnées qui s'appliquent tous les deux aux manuscrits et aux imprimés. En plus de cela, il faut aussi chercher les métadonnées qui, selon l'usage de la \acrshort{BNF}, sont normalement encodées dans le \textit{manifest} \acrshort{IIIF} d'un document sur Gallica. J'ai déterminé que l'essentiel du \textit{manifest} \acrshort{IIIF} sont les métadonnées listées dans le Tableau~\ref{tab:metadataManfiest}.

\begin{figure}[htp]
\begin{tabularx}{\textwidth}{ 
  | m{5em}
  | >{\arraybackslash}X 
  |}
\hline
\textbf{\textit{label}} & \textbf{\textit{value}}\\
\hline \hline
Relation & Notice du catalogue : http://catalogue.bnf.fr/ark:/12148/cb33352789b \\ \hline
Repository & Bibliothèque nationale de France\\ \hline
Shelfmark & Bibliothèque nationale de France, département Réserve des livres rares, RES-YF-1990\\ \hline
Title & Les divers entretiens de la fontaine de Vaucluse~: ballet dansé à la grande salle du Roure l'année 1649\\ \hline
Language & français\\ \hline
Creator & Nouguier, de. Auteur du texte\\ \hline
Date & 1649\\ \hline
\end{tabularx}
\caption{Les métadonnées essentielles du \textit{manifest} \acrshort{IIIF} (\acrshort{ARK} \texttt{bpt6k1513919s})}
\label{tab:metadataManfiest}
\end{figure}

Certaines métadonnées du \textit{manifest} posent des avantages et des défis. L'un des avantages est la manière par laquelle la \acrshort{BNF} encode la donnée \textit{Relation} pour les documents mis sur Gallica. La donnée \textit{Relation} peut se répéter parce que le fac-similé peut être lié aux plusieurs notices du catalogue. Cependant, elle termine toujours par l'\acrshort{ARK} unique du document physique auquel le fac-similé est lié. Ainsi, l'application \texttt{alto2tei} dispose de la donnée \textit{Relation} et, en traitant la fin de la chaîne, extrait l'\acrshort{ARK} du document physique dans le catalogue général de la \acrshort{BNF}. Un défi à surmonter est quand un type (\textit{label}) de métadonnée n'est pas répété mais il possède plusieurs valeurs. Cela peut arriver quand un document a plusieurs auteurs. Pour cette raison, l'application \texttt{alto2tei} traite l'encodage de toute donnée portant sur l'auteur d'une manière différente que celle pour les autres métadonnées.

\subsection{Le format \Gls{unimarc}}
Étant communiquées en \acrshort{XML}, les données du catalogue général de la \acrshort{BNF} se trouvent sur l'arbre de la notice en format \Gls{unimarc}~; ce format leur donne le préfix \texttt{mxc} qui fait référence au \textit{MarcXchange}. Un exemple des données \Gls{unimarc} renvoyées par l'\acrshort{API} \acrshort{SRU} de la \acrshort{BNF} est montré dans la Figure~\ref{fig:unimarc}. L'exemple prend toujours le livret imprimé du ballet \textit{Les divers entretiens de la fontaine de Vaucluse} de 1649. Le fac-similé numérique sur Gallica a l'identifiant \texttt{bpt6k1513919s}. Mais, comme expliqué ci-dessus, l'identifiant du document physique est récupéré en examinant le \textit{manifest} \acrshort{IIIF} du fac-similé numérique. L'\acrshort{ARK} du document physique (\texttt{cb33352789b} dans l'exemple) permet d'interroger les données du catalogue général de la \acrshort{BNF} grâce à son \acrshort{API} \textit{\acrlong{SRU}} (\acrshort{SRU}).

\begin{figure}[ht]

\begin{subfigure}[b]{\textwidth}
\lstset{style=python}
\begin{lstlisting}[language=python]
class SRU:
	def __init__(self, ark):
		"""Args: 
			ark (string): document ARK in BnF catalogue"""
		self.ark = ark
		
	def request(self):
		r = requests.get(f'http://catalogue.bnf.fr/api/SRU?version=1.2&operation=searchRetrieve&query=(bib.persistentid all "ark:/12148/{self.ark}")')
\end{lstlisting}
\caption{La requête envoyée à l'\acrshort{API} \acrshort{SRU} par l'application python \texttt{alto2tei}}
\label{fig:sruRequest}
\end{subfigure}

\begin{subfigure}[b]{\textwidth}
\begin{lstlisting}[language=XML]
<srw:searchRetrieveResponse xmlns:srw="http://www.loc.gov/zing/srw/" xmlns="http://catalogue.bnf.fr/namespaces/InterXMarc">
<!-- ... -->
	<srw:echoedSearchRetrieveRequest>
		<srw:query>(bib.persistentid all "ark:/12148/cb33352789b")</srw:query>
	</srw:echoedSearchRetrieveRequest>
	<srw:numberOfRecords>1</srw:numberOfRecords>
	<srw:records>
		<srw:record>
			<srw:recordSchema>marcxchange</srw:recordSchema>
			<srw:recordPacking>xml</srw:recordPacking>
			<srw:recordData>
				<mxc:record format="Unimarc" id="ark:/12148/cb33352789b" type="Bibliographic" xmlns:mxc="info:lc/xmlns/marcxchange-v2">
<!-- data in Unimarc formatting, with prefix "mxc" -->
				</mxc:record>
			</srw:record>
		</srw:recordData>
<!-- ... -->
\end{lstlisting}
\caption{La réponse de l'\acrshort{API} \acrshort{SRU} pour le document d'\acrshort{ARK} \texttt{bpt6k1513919s}}
\label{fig:unimarc}
\end{subfigure}

\caption{La requête et la réponse à l'API SRU [version 1.2]}
\end{figure}

Les données \Gls{unimarc} sont emballées dans les éléments du préfix \texttt{srw} qui fait référence au protocole \acrshort{SRU}\footcite{SRUSearchRetrieval}. L'élément \texttt{<srw:recordData>} (lignes 11-15, Fig~\ref{fig:unimarc}) présente les données da la notice du catalogue trouvée. L'application \texttt{alto2tei} détermine qu'elle a bien localisé le bon document physique dans le catalogue général de la \acrshort{BNF} si la réponse de l'\acrshort{API} \acrshort{SRU} ne contient qu'une seule notice. Le nombre de notices trouvées est sur la ligne 6 de la Figure~\ref{fig:unimarc}. Si ce nombre est plus que 1 cela indique que l'\acrshort{ARK} récupérée depuis la donnée \textit{Relation} du \textit{manifest} \acrshort{IIIF} est liée aux notices de plusieurs documents. Ne pouvant pas déterminer quel document est le bon, l'application \texttt{alto2tei} abandonne les données \Gls{unimarc} et compte uniquement sur les métadonnées du \textit{manifest} \acrshort{IIIF}.

\subsection{Le format des données du site \acrshort{SUDOC}}
Puisque les données \Gls{unimarc} ne renseignent pas sur le lieu de l'institution qui conserve le document physique, il faut d'autre source de données pour remplir l'élément \acrshort{TEI} \texttt{<settlement>} du \texttt{<sourceDesc>}. Pour trouver une solution, j'en ai discuté avec une spécialiste à la \acrlong{BNF}, Florence Tfibel. Dans un courriel, Tfibel a dit que \og{}la localisation de l'institution n'y est effectivement nulle part renseignée \textit{en dur}\fg{} dans les données \Gls{unimarc} du catalogue\footcite{tfibelRETrGallic2022}. Alors elle a eu l'idée de la chercher sur le site web du \acrlong{SUDOC} qui gère un répertoire des centres de ressources (RCR), auquel appartient la \acrshort{BNF}. La \acrshort{BNF} dispose d'un numéro RCR. Son emplacement sur l'arbre des données \Gls{unimarc} du catalogue est montré dans la Figure~\ref{fig:rcr}. Le numéro RCR de la \acrshort{BNF} est 759999999.
\begin{figure}[ht]
\begin{lstlisting}[language=XML]
<mxc:datafield tag="930" ind1=" " ind2=" ">
<!-- ... -->
	<mxc:subfield code="b">759999999</mxc:subfield>
<!-- ... -->
</mxc:datafield>
\end{lstlisting}
\caption{L'emplacement du RCR sur l'arbre des données \Gls{unimarc}}
\label{fig:rcr}
\end{figure}

Le site web du \acrlong{SUDOC} ne dispose pas d'un genre d'\acrshort{API} \acrshort{SRU} aussi efficace que celui de la \acrshort{BNF}, mais les données de son site peuvent être interrogées en traitant la page du centre de ressource qui porte le numéro cherché. Il est possible de lire la page destinée à l'exposition d'un centre de recherche, telle que la \acrshort{BNF}, en construisant l'URL que le site attend à la suite d'une recherche. Si on recherche le numéro de la \acrshort{BNF} sur le site de \acrshort{SUDOC}, l'URL sera la suivante, où le dernier chiffre après la signe d'égalité est le numéro RCR de la \acrshort{BNF}~:
\begin{verbatim}
http://www.sudoc.abes.fr/cbs/xslt//DB=2.2/
    CMD?ACT=SRCHA&IKT=8888&SRT=RLV&TRM=759999999
\end{verbatim}
L'application \texttt{alto2tei} demande à l'\acrshort{API} du site \acrshort{SUDOC} la page de cette URL~; ensuite, elle traite les données \acrshort{HTML} renvoyées, voir Fig~\ref{fig:sudoc}. La page \acrshort{HTML} possède un tableau avec une une cellule (\texttt{<td>}) qui contient la phrase \og{}Adresse postale\fg{} dans le \texttt{<span>} d'un \texttt{<div>}. Après cette cellule il y a une autre cellule avec les données portant sur l'adresse du centre. Ayant l'a localisée avec l'aide de la première cellule, l'application \texttt{alto2tei} traite les données de cette dernière cellule et extrait la ville et le nom uniforme du centre. Même si la ville et le nom de la \acrshort{BNF} seront toujours Paris et la \acrlong{BNF}, respectivement, il est meilleur de s'appuyer sur les sources de données pour que l'application puisse s'adapter plus facilement aux autres utilisations.

\begin{figure}[ht]
\begin{lstlisting}[language=XML]
<tr>
	<td class="rec_lable">
		<div>
			<span>Adresse postale : </span>
		</div>
	</td>
	<td clas="rec_title">
		<div>
			<span>Bibliothèque nationale de France</span>
		</div>
		<div>
			<span>quai Francois Mauriac </span>
		</div>
		<div>
			<span>75706 Paris CEDEX 13</span>
		</div>
		<div>
			<span>France</span>
		</div>
		<div>
			<span> </span>
		</div>
	</td>
</tr>
\end{lstlisting}
\caption{Les données cherchées depuis le site \acrshort{SUDOC}}
\label{fig:sudoc}
\end{figure}

\subsection{Le format du fichier de configuration}
Une partie importante des métadonnées s'appui aussi sur le fichier de configuration que les utilisateurs de l'application \texttt{alto2tei} peuvent facilement personnaliser. Son format est le \acrshort{YAML} (\textit{\acrlong{YAML}}) qui n'exige pas de grande expertise dans la programmation. Comme \textit{l'\acrlong{XML}} (\acrshort{XML}) le \acrshort{YAML} est ce qu'on appelle un langage de \og{}Mark Up\fg{} et donc il imbrique des données. Mais au lieu d'utiliser les éléments \acrshort{XML} (\texttt{< >}), le \acrshort{YAML} utilise simplement les deux points et de la tabulation. Ainsi les utilisateurs de divers spécialités peuvent adapter l'application \texttt{alto2tei} et le pipeline \textit{Gallic(orpor)a} aux données de leur propre projet.

Le fichier de configuration informe trois aspects de la création du document \acrshort{TEI}. Dans un premier temps, il informe l'application \texttt{alto2tei} où il peut trouver les fichiers \acrshort{ALTO} qui a été traités par les modèles \acrshort{HTR}. Cette donnée est personnalisable pour qu'une utilisatrice ou un utilisateur puisse transformer les transcriptions \acrshort{ALTO} stockés sous divers dossiers. Dans un deuxième temps, le fichier de configuration informe la requête que l'application \texttt{alto2tei} envoie à l'\acrshort{API} \acrshort{IIIF}, la même \acrshort{API} qui réserve d'accès aux images et aux métadonnées du fac-similé numérique et qui est discutée dans la section~\ref{sec:manifest}. L'application \texttt{alto2tei} permet de personnaliser la requête envoyée à l'\acrshort{API} \acrshort{IIIF} selon les normes de l'institution qui la gère. Le fichier configuré pour l'\acrshort{API} \acrshort{IIIF} gérée par la \acrshort{BNF} est montré dans la Figure~\ref{fig:configIIIF}. Les paramètres \texttt{manifest\_prefix} et \texttt{image\_prefix} sont pareils pour l'\acrshort{API} \acrshort{IIIF} de la \acrshort{BNF}, mais le fichier de configuration les garde distinct afin qu'ils puissent s'adapter aux APIs qui distinguent entre l'URL de l'\Gls{iiifapi} et de la Presentation \acrshort{API}.

\begin{figure}[ht]
\lstset{style=python}
\begin{lstlisting}[language=python]
iiifURI:
	scheme: "https"
	server: "gallica.bnf.fr"
	manifest_prefix: "/iiif/ark:/12148/"
	image_prefix: "/iiif/ark:/12148/"
	manifest_suffix: "/manifest.json"
\end{lstlisting}
\caption{La partie du fichier de configuration portant sur la requête envoyée à l'\acrshort{API} \acrshort{IIIF}}
\label{fig:configIIIF}
\end{figure}

Enfin, le fichier de configuration informe les éléments descendant du \texttt{<publicationStmt>} dans le \texttt{<fileDesc>} du \texttt{<teiHeader>}. Les métadonnées récupérées par l'application \texttt{alto2tei} ainsi que leur format en \acrshort{YAML} sont montrés dans la Figure~\ref{fig:configResp}. Le nom du projet (ligne 18, Fig.~\ref{fig:configResp}) et les noms des membres de l'équipe peuvent être personnalisés. Le prénom ou les prénoms de chaque membre sont renseignés par la chaîne entre guillemets qui suit les deux points après le terme \textit{forename}. Le terme \textit{surname} indique le même chose pour le nom de famille. L'identifiant du membre peut être personnalisé soit en le type d'identifiant, tel que ORCID, soit en la valeur qui suit les deux points après le terme \textit{target}.

\begin{figure}[ht]
\lstset{style=python}
\begin{lstlisting}[language=python]
responsibility:
	text: "Transformation from ALTO4 to TEI by"
	resp: [{"forename":"Kelly", 
		"surname":"Christensen", 
		"ptr":{"type":"orcid","target":"000000027236874X"}},
		
		{"forename":"Simon",
		"surname":"Gabay", 
		"ptr":{"type":"orcid","target":"0000000190944475"}},
		
		{"forename":"Ariane",
		"surname":"Pinche",
		"ptr":{"type":"orcid","target":"0000000278435050"}}
		]
	# name the publisher of this digital edition (project name)
	publisher: "Gallic(orpor)a"
	# name which authority finances or supports the project
	authority: "BnF DATALab"
	# specify the edition's availability to the public/licencing
	availability: {"status": "restricted", "n":"cc-by"}
	licence: {"target":"https://creativecommons.org/licenses/by/4.0/"}
\end{lstlisting}
\caption{La partie du fichier de configuration portant sur la requête envoyée à l'\acrshort{API} \acrshort{IIIF}}
\label{fig:configResp}
\end{figure}

\section{La mise en place des données récupérées}
Ayant découvert les structures de données des sources de données externes, il faut parler d'où ses données ainsi récupérées vont dans le \texttt{<teiHeader>}. Les métadonnées intégrées au document \acrshort{TEI} s'appuient sur (1) le \textit{manifest} \acrshort{IIIF} envoyé par l'\acrshort{API} \acrshort{IIIF} en format \acrshort{JSON}, (2) les données \Gls{unimarc} du catalogue général de la \acrshort{BNF} envoyées par l'\acrshort{API} \acrshort{SRU} en format \acrshort{XML}, (3) les données \acrshort{HTML} récupérées à partir de la page de recherche du site \acrshort{SUDOC}, et (4) le fichier de configuration en format \acrshort{YAML}. L'application recherche des données dans deux autres sources~: (5) les fichiers \acrshort{XML} \acrshort{ALTO} et (6) la commande saisie pour démarrer l'application. Dans un premier temps, l'application \texttt{alto2tei} compte les fichiers \acrshort{ALTO} du fac-similé numérique et utilise ce nombre pour l'élément \texttt{<extent>}. Dans un deuxième temps, la commande saisie pour la démarrer exige un paramètre qui déclare la version de l'engin \textit{Kraken} qui a été utilisée pour mettre en œuvre les modèles \acrshort{HTR} et réaliser les transcriptions.

\subsection{Les sources des données du \texttt{<titleStmt>}}
La Figure~\ref{fig:titleStmt} visualise les sources de données du \texttt{<titleStmt>}. Les éléments portant sur le titre de la ressource (\texttt{<title>}) et sur les personnes auxquels est attribuée la propriété intellectuelle du texte représenté (\texttt{<author>}) sont informés par l'un de deux sources de données possibles. Soit les données du catalogue général donnent beaucoup d'information sur les auteurs ainsi qu'un titre propre, soit le \textit{manifest} \acrshort{IIIF} donne simplement le titre du texte et les noms non annotés des auteurs. Le choix entre les deux sources de données se fait à partir de la disponibilité des données du catalogue de la \acrshort{BNF}. Normalement, l'application \texttt{alto2tei} met en priorité les données du catalogue.

La Figure~\ref{fig:sourceAuthor} visualise les éléments du \texttt{<titleStmt>} dans le cas où la relation entre le fac-similé numérique sur Gallica et le document physique appartenant à la \acrshort{BNF} est établie. Cela veut dire que le document dont parle la notice du catalogue est le même document à partir duquel le fac-similé numérique sur Gallica a été fait. Quand la notice du bon document est trouvée, les données \Gls{unimarc} cherchées depuis le catalogue sont (1) le titre uniforme du document et (2) le nom et l'identifiant de ses auteurs. Les éléments descendants du \texttt{<respStmt>} s'appuient toujours sur le fichier de configuration afin qu'ils puissent être personnalisées par les personnes qui utilisent l'application \texttt{alto2tei} pour générer leurs propres documents \acrshort{TEI}. Toutes ces sources ainsi que la destination de leurs données sont visualisées dans la Figure~\ref{fig:titleStmt}. Les Tableaux~\ref{tab:author} et \ref{tab:titleStmt} décrivent d'où viennent chaque donnée des sources de données en \acrshort{JSON} et \acrshort{XML}, soit le \textit{manifest} \acrshort{IIIF} et le catalogue général, respectivement\footnote{Pour les attributs d'un élément \acrshort{XML}, la table utilise le syntaxe \textit{XPath} par lequel l'attribut est déclaré avec l'arobase. Par exemple, l'attribut \texttt{@att} d'un élément \texttt{<element>}, qui descende de l'élément \texttt{<parent>}, serait représenté dans le format suivant~: \texttt{parent/element[@att]}.}

\tikzstyle{empty} = [rectangle, rounded corners, minimum width=2cm, minimum height=0.5cm]
\tikzstyle{iiif} = [rectangle, rounded corners, minimum width=2cm, minimum height=0.5cm, text centered, draw=black, fill=blue!30]
\tikzstyle{attIIIF} = [rectangle, rounded corners, minimum width=2cm, minimum height=0.5cm, text centered, draw=black, fill=blue!30, dashed]
\tikzstyle{unimarc} = [rectangle, rounded corners, minimum width=2cm, minimum height=0.50cm, text centered, draw=black, fill=red!30]
\tikzstyle{attUnimarc} = [rectangle, rounded corners, minimum width=2cm, minimum height=0.50cm, text centered, draw=black, fill=red!30, dashed]
\tikzstyle{config} = [rectangle, rounded corners, minimum width=2cm, minimum height=0.5cm, text centered, draw=black, fill=green!30]
\tikzstyle{attConfig} = [rectangle, rounded corners, minimum width=2cm, minimum height=0.5cm, text centered, draw=black, fill=green!30, dashed]
\tikzstyle{sudoc} = [rectangle, rounded corners, minimum width=2cm, minimum height=0.5cm, text centered, draw=black, fill=yellow!30]
\tikzstyle{attSudoc} = [rectangle, rounded corners, minimum width=2cm, minimum height=0.5cm, text centered, draw=black, fill=yellow!30, dashed]
\tikzstyle{files} = [rectangle, rounded corners, minimum width=2cm, minimum height=0.5cm, text centered, draw=black, fill=orange!30]
\tikzstyle{attFiles} = [rectangle, rounded corners, minimum width=2cm, minimum height=0.5cm, text centered, draw=black, fill=orange!30, dashed]
\tikzstyle{input} = [rectangle, rounded corners, minimum width=2cm, minimum height=0.5cm, text centered, draw=black, fill=violet!30]
\tikzstyle{attInput} = [rectangle, rounded corners, minimum width=2cm, minimum height=0.5cm, text centered, draw=black, fill=violet!30, dashed]
\tikzstyle{elem} = [rectangle, rounded corners, minimum width=2cm, minimum height=0.5cm, text centered, draw=black]
\tikzstyle{att} = [rectangle, rounded corners, minimum width=2cm, minimum height=0.5cm, text centered, draw=black, dashed]
\tikzstyle{elemDefault} = [rectangle, rounded corners, minimum width=2cm, minimum height=0.5cm, text centered, draw=black, fill=gray!30]
\tikzstyle{attDefault} = [rectangle, rounded corners, minimum width=2cm, minimum height=0.5cm, text centered, draw=black, fill=gray!30, dashed]
\tikzstyle{arrow} = [thick,>=stealth]
\tikzset{line/.style={draw, thick, -latex'}}

\begin{figure}[htp]
\centering
% legend
\begin{subfigure}[b]{0.4\textwidth}
\begin{tikzpicture}
\matrix [draw] {
	\node [iiif,label=right:manifest IIIF] {};\\
	\node [unimarc,label=right:Unimarc] {};\\
	\node [sudoc,label=right:SUDOC] {}; \\
	\node [config,label=right:fichier de config] {}; \\
	\node [files,label=right:fichiers traités] {}; \\
	\node [input,label=right:saisie] {}; \\
	\node [elemDefault,label=right:par défaut] {}; \\
};
\end{tikzpicture}
\end{subfigure}

\vspace{1cm}


% <author> with relation
\begin{subfigure}[b]{\textwidth}\centering
\begin{tikzpicture}[node distance=0.5cm]

\node [elem] (fileDesc) {fileDesc};
\node [elem, right=of fileDesc] (titleStmt) {titleStmt};
\node [unimarc, right=of titleStmt] (title) {title};
\node [elem, below=of title] (author) {author};
\node [elem, right=of author] (persName) {persName};
\node [unimarc, right=of persName] (forename) {forename};
\node [unimarc, below=of forename] (surname) {surname};
\node [elem, below=of surname] (ptr) {ptr};
\node [attUnimarc, right=of ptr] (isni) {@type};
\node [attUnimarc, below=of isni] (target) {@target};

\draw [arrow] (fileDesc) -- (titleStmt);
\draw [arrow] (titleStmt) -- (title);
\draw [arrow] (author) -| (titleStmt);
\draw [arrow] (author) -- (persName);
\draw [arrow] (persName) -- (forename);
\draw [arrow] (surname) -| (persName);
\draw [arrow] (ptr) -| (persName);
\draw [arrow, dashed] (ptr) -- (isni);
\draw [arrow, dashed] (ptr) |- (target);

\end{tikzpicture}
\caption{Les sources de données du \texttt{<title>} et du \texttt{<author>} quand la relation entre le fac-similé numérique et le document physique est établie}
\label{fig:sourceAuthor}
\end{subfigure}

\vspace{1cm}

% <author> without relation
\begin{subfigure}[b]{\textwidth}\centering
\begin{tikzpicture}[node distance=0.5cm]

\node [elem] (fileDesc) {fileDesc};
\node [elem, right=of fileDesc] (titleStmt) {titleStmt};
\node [iiif, right=of titleStmt] (title) {title};
\node [elem, below=of title] (author) {author};
\node [elem, right=of author] (persName) {persName};
\node [iiif, right=of persName] (name) {name};

\draw [arrow] (fileDesc) -- (titleStmt);
\draw [arrow] (titleStmt) -- (title);
\draw [arrow] (author) -| (titleStmt);
\draw [arrow] (author) -- (persName);
\draw [arrow] (persName) -- (forename);
\draw [arrow] (persName) -- (name);

\end{tikzpicture}
\caption{Les sources de données du \texttt{<title>} et du \texttt{<author>} quand la relation entre le fac-similé numérique et le document physique \textit{n'est pas} établie}
\label{fig:sourceAuthorWO}
\end{subfigure}

\vspace{1cm}

% <respStmt>
\begin{subfigure}[b]{\textwidth}\centering
\begin{tikzpicture}[node distance=0.5cm]

\node [elem] (fileDesc) {fileDesc};
\node [elem, right=of fileDesc] (titleStmt) {titleStmt};
\node [elem, right=of titleStmt] (respStmt) {respStmt};
\node [config, right=of respStmt] (resp) {resp};
\node [elem, right=of respStmt, below=of resp] (persName) {persName};
\node [config, right=of persName] (forename) {forename};
\node [config, below=of forename] (surname) {surname};
\node [elem, below=of surname] (ptr) {ptr};
\node [attConfig, right=of ptr] (orcid) {@type};
\node [attConfig, below=of orcid] (target) {@target};

\draw [arrow] (fileDesc) -- (titleStmt);
\draw [arrow] (titleStmt) -- (respStmt);
\draw [arrow] (respStmt) -- (resp);
\draw [arrow] (persName) -| (respStmt);
\draw [arrow] (persName) -- (forename);
\draw [arrow] (surname) -| (persName);
\draw [arrow] (ptr) -| (persName);
\draw [arrow, dashed] (ptr) -- (orcid);
\draw [arrow, dashed] (ptr) |- (target);
	
\end{tikzpicture}
\caption{Les sources de données du \texttt{<respStmt>} dans tout cas}
\label{fig:titleStmt}
\end{subfigure}

\vspace{1cm}

\caption{Les sources de données des éléments du \texttt{<titleStmt>}}
\label{fig:titleStmt}
\end{figure}

\begin{figure}[htp]
\centering
\begin{tabularx}{\textwidth}
{	| >{\arraybackslash}X 
	| >{\arraybackslash}X 
	| >{\arraybackslash}X |}
\hline
élément ou attribut \acrshort{TEI} & \textit{manifest} \acrshort{IIIF} en \acrshort{JSON} & données \acrshort{XML} \Gls{unimarc} \\
\hline \hline
\texttt{author/} \texttt{name} & \texttt{"Creator"} & --\\ \hline
\texttt{author/[@xml:id]} & 2 premières caractères de la valeur \texttt{"Creator"} + chiffre & 2 premières caractères de la valeur \texttt{datafield[@tag="700"]/} \texttt{subfield[@code="a"]}  ou \texttt{datafield[@tag="703"]/} \texttt{subfield[@code="a"]}  + chiffre \\ \hline
\texttt{persName/} \texttt{forename} & -- & \texttt{datafield[@tag="700"]/} \texttt{subfield[@code="b"]} ou \texttt{datafield[@tag="703"]/} \texttt{subfield[@code="b"]} sans les éléments suivants: \textit{van der, de la, de, du, von, van} \\ \hline
\texttt{persName/} \texttt{namelink} & -- & l'un des éléments \textit{van der, de la, de, du, von,} ou \textit{van} extrait du \texttt{datafield[@tag="700"]/} \texttt{subfield[@code="b"]} ou \texttt{datafield[@tag="703"]/} \texttt{subfield[@code="b"]} \\ \hline
\texttt{persName/} \texttt{surname} & -- & \texttt{datafield[@tag="700"]/} \texttt{subfield[@code="a"]}  ou \texttt{datafield[@tag="703"]/} \texttt{subfield[@code="a"]}\\ \hline\
\texttt{persName/} \texttt{ptr[@target]} & -- & \texttt{datafield[@tag="700"]/} \texttt{subfield[@code="o"]}  ou \texttt{datafield[@tag="703"]/} \texttt{subfield[@code="o"]} \\ \hline

\end{tabularx}
\caption{\textit{Mapping} des données du \texttt{<author>} utilisé dans le \texttt{<titleStmt>} et le \texttt{<sourceDesc>}}
\label{tab:author}
\end{figure}

\begin{figure}[htp]\centering
\begin{tabularx}{\textwidth}
{	| >{\arraybackslash}X 
	| >{\arraybackslash}X 
	| >{\arraybackslash}X |}
\hline
élément ou attribut \acrshort{TEI} & \textit{manifest} \acrshort{IIIF} & données \Gls{unimarc}\\
\hline \hline
\texttt{titleStmt/} \texttt{title} & \texttt{"Title"} & \texttt{datafield[@tag="200"]/} \texttt{subfield[@code="a"]} \\ \hline

\end{tabularx}
\caption{\textit{Mapping} des données du \texttt{<titleStmt>}}
\label{tab:titleStmt}
\end{figure}

\subsection{Les sources des données du \texttt{<extent>}}
L'élément \texttt{<extent>} du \texttt{<teiHeader>}, qui indique le nombre de fichiers traités par l'application \texttt{alto2tei} lors de la création de la ressource numérique, s'appuie simplement sur un compte des fichiers. L'application a besoin de traiter tout fichier \acrshort{XML} \acrshort{ALTO} dans le dossier du document, comme expliqué dans la section~\ref{structure-fichier}, et mettre ses données dans le \texttt{<sourceDoc>}, selon la modélisation de l'application \texttt{alto2tei}, voir la Figure~\ref{fig:dataExtent}. Elle peut également compter le nombre des fichiers \acrshort{XML} \acrshort{ALTO} qui appartiennent au document. Ce compte elle met comme la \og{}taille\fg{} de la ressource numérique, dans l'élément \texttt{<measure>}.

\begin{figure}[htp]
\centering
% legend
\begin{subfigure}[b]{\textwidth}\centering
\begin{tikzpicture}
\matrix [draw] {
	\node [iiif,label=right:manifest IIIF] {};\\
	\node [unimarc,label=right:Unimarc] {};\\
	\node [sudoc,label=right:SUDOC] {}; \\
	\node [config,label=right:fichier de config] {}; \\
	\node [files,label=right:fichiers traités] {}; \\
	\node [input,label=right:saisie] {}; \\
	\node [elemDefault,label=right:par défaut] {}; \\
};
\end{tikzpicture}
\end{subfigure}

\vspace{0.5cm}

\begin{subfigure}[b]{\textwidth}\centering
\begin{tikzpicture}[node distance=0.5cm]
\node [elem] (fileDesc) {fileDesc};
\node [elem, right=of fileDesc] (start) {extent};
\node [elem, right=of start] (measure) {measure};
\node [attDefault, right=of measure] (unit) {@unit};
\node [attFiles, below=of unit] (n) {@n};

\draw [arrow] (fileDesc) -- (start);
\draw [arrow] (start) -- (measure);
\draw [arrow, dashed] (measure) -- (unit);
\draw [arrow, dashed] (measure) |- (n);

\end{tikzpicture}
\end{subfigure}

\caption{La source des données du \texttt{<extent>}}
\label{fig:dataExtent}
\end{figure}

\subsection{Les sources des données du \texttt{<publicationStmt>}}
La distribution de la ressource numérique est décrite par le \texttt{<publicationStmt>}. La Figure~\ref{fig:dataPS} montre comment toute métadonnée du \texttt{<publicationStmt>} vient du fichier de configuration, où les utilisateurs du pipeline et/ou de l'application peuvent personnaliser les détails de leur projet et les droits d'utilisation.

\begin{figure}[htp]
\centering

\begin{tikzpicture}[node distance=0.5cm]

\node [elem] (fileDesc) {fileDesc};
\node [elem, right=of fileDesc] (publicationStmt) {publicationStmt};
\node [config, right=of publicationStmt] (publisher) {publisher};
\node [config, below=of publisher] (authority) {authority};
\node [elem, below=of authority] (availability) {availability};
\node [attConfig, right=of availability] (status) {@status};
\node [attConfig, below=of status] (n) {@n};
\node [empty, left=of n] (empty) {};
\node [elem, below=of empty] (licence) {licence};
\node [attConfig, right=of licence] (target) {@target};
\node [elem, below=of licence] (date) {date};
\node [attFiles, right=of date] (when) {@when};

\draw [arrow] (fileDesc) -- (publicationStmt);
\draw [arrow] (publicationStmt) -- (publisher);
\draw [arrow] (publicationStmt) |- (authority);
\draw [arrow] (publicationStmt) |- (availability);
\draw [arrow] (publicationStmt) |- (date);
\draw [arrow, dashed] (availability) -- (status);
\draw [arrow, dashed] (availability) |- (n);
\draw [arrow] (publicationStmt) |- (licence);
\draw [arrow, dashed] (licence) -- (target);
\draw [arrow] (publicationStmt) |- (date);
\draw [arrow, dashed] (date) -- (when);

\end{tikzpicture}


\caption{La source des données du \texttt{<publicationStmt>}}
\label{fig:dataPS}
\end{figure}


\subsection{Les sources des données du \texttt{<sourceDesc>}}

Le \texttt{<sourceDesc>} décrit le document source dont la transcription et le texte sont représentés par la ressource numérique. Il se compose de deux éléments principales. Dans un premier temps, le \texttt{<sourceDesc>} contient le \texttt{<bibl>} qui est une description de la source, portant sur les auteurs, le titre, l'éditeur, la date et le lieu de création. Dans un deuxième temps, le \texttt{<sourceDesc>} contient le \texttt{<msDesc>} qui est une citation bibliographique du document source. Si besoin, l'application \texttt{alto2tei} peut livrer un \texttt{<sourceDesc>} en utilisant uniquement le \textit{manifest} \acrshort{IIIF}. Mais il n'est pas une source sûre sur l'éditeur ni sur le lieu de publication du document source. Les données \Gls{unimarc} du catalogue général de la \acrshort{BNF}, par contre, portent toujours sur ces détails. Même si le catalogue général dit que la donnée n'est pas connue, la notice renseigne toujours sur l'éditeur et le lieu de publication du document et l'application \texttt{alto2tei} répète l'information incertaine du catalogue. Les Tableaux~\ref{tab:sourceDescBibl} et \ref{tab:sourceDescMS} précisent quelles données sont intégrées dans le \texttt{<sourceDesc>} et lesquelles sont laissées vides si la relation entre le fac-similé numérique et le document physique n'est pas établie. Les Figures~\ref{fig:dataBibl} et \ref{fig:dataMsDesc} visualisent les source de données pour ces deux éléments selon leur disponibilité. 

\begin{figure}[htp]\centering
\begin{tabularx}{\textwidth}
{	| >{\arraybackslash}X 
	| >{\arraybackslash}X 
	| >{\arraybackslash}X |}
\hline
élément ou attribut \acrshort{TEI} & \textit{manifest} \acrshort{IIIF} & données \Gls{unimarc}\\
\hline \hline
\texttt{sourceDesc/} \texttt{bibl/} \texttt{ptr[@target]} & -- & \texttt{controlfield[@tag="003"]} \\ \hline
\texttt{sourceDesc/} \texttt{bibl/} \texttt{title} & \texttt{"Title"} & \texttt{datafield[@tag="200"]/} \texttt{subfield[@code="a"]} \\ \hline
\texttt{sourceDesc/} \texttt{bibl/} \texttt{pubPlace} & -- & \texttt{datafield[@tag="210"]/} \texttt{subfield[@code="a"]} \\ \hline
\texttt{sourceDesc/} \texttt{bibl/} \texttt{pubPlace[@key]} & -- & \texttt{datafield[@tag="102"]/} \texttt{subfield[@code="a"]} \\ \hline
\texttt{sourceDesc/} \texttt{bibl/} \texttt{publisher} & -- & \texttt{datafield[@tag="210"]/} \texttt{subfield[@code="c"]} \\ \hline
\texttt{sourceDesc/} \texttt{bibl/} \texttt{date} & \texttt{"Date"} & Caractères 9-13 de \texttt{datafield[@tag="100"]/} \texttt{subfield[@code="a"]} \\ \hline
\texttt{sourceDesc/} \texttt{bibl/} \texttt{date[@when} & -- & Caractères 9-13 de \texttt{datafield[@tag="100"]/} \texttt{subfield[@code="a"]} \\ \hline
\texttt{sourceDesc/} \texttt{bibl/} \texttt{date[@cert} & -- & 8\textsuperscript{e} caractère de \texttt{datafield[@tag="100"]/} \texttt{subfield[@code="a"]} \\ \hline
\end{tabularx}
\caption{\textit{Mapping} des données du \texttt{<bibl>} du \texttt{<sourceDesc>}}
\label{tab:sourceDescBibl}
\end{figure}

%%%%%%%%%%%%%%%%%%
\begin{figure}[htp]
\centering
% legend
\begin{subfigure}[b]{\textwidth}\centering
	\begin{tikzpicture}[node distance=1cm]
		\matrix [draw] {
			\node [iiif,label=right:manifest IIIF] {};\\
			\node [unimarc,label=right:Unimarc] {};\\
			\node [sudoc,label=right:SUDOC] {}; \\
			\node [config,label=right:fichier de config] {}; \\
			\node [files,label=right:fichiers traités] {}; \\
			\node [input,label=right:saisie] {}; \\
			\node [elemDefault,label=right:par défaut] {}; \\
		};
	\end{tikzpicture}
\end{subfigure}

\vspace{1cm}

% <bibl> IIIF
\begin{subfigure}[b]{\textwidth}\centering
	\begin{tikzpicture}[node distance=0.25cm]

		\node [elem] (fileDesc) {fileDesc};
		\node [elem, right=of fileDesc] (sourceDesc) {sourceDesc};
		\node [elem, right=of sourceDesc] (bibl) {bibl};
		\node [elem, right=of bibl] (ptr) {ptr (\textit{vide})};
		\node [elem, below=of ptr] (author) {author};
		\node [iiif, right=of author] (name) {name};
		\node [iiif, below=of author] (title) {title};
		\node [elem, below=of title] (pubPlace) {pubPlace (\textit{vide})};
		\node [elem, below=of pubPlace] (publisher) {publisher (\textit{vide})};
		\node [iiif, below=of publisher] (date) {date};
		
		\draw [arrow] (fileDesc) -- (sourceDesc);
		\draw [arrow] (sourceDesc) -- (bibl);
		\draw [arrow] (bibl) -- (ptr);
		\draw [arrow] (bibl) |- (author);
		\draw [arrow] (author) -- (name);
		\draw [arrow] (bibl) |- (title);
		\draw [arrow] (bibl) |- (pubPlace);
		\draw [arrow] (bibl) |- (publisher);
		\draw [arrow] (bibl) |- (date);

	\end{tikzpicture}
	
\caption{Quand la relation entre le fac-similé numérique (renseigné par le \textit{manifest} \acrshort{IIIF} et le document physique (renseigné par catalogue général de la \acrshort{BNF}) n'est pas établie}
\end{subfigure}

\vspace{1cm}
% <bibl> SRU
\begin{subfigure}[b]{\textwidth}\centering
	\begin{tikzpicture}[node distance=0.25cm]

		\node [elem] (fileDesc) {fileDesc};
		\node [elem, right=of fileDesc] (sourceDesc) {sourceDesc};
		\node [elem, right=of sourceDesc] (bibl) {bibl};
		\node [elem, right=of bibl] (ptr) {ptr};
		\node [attUnimarc, right=of ptr] (target) {@target};
		\node [elem, below=of ptr] (author) {author};
		\node [elem, right=of author] (persName) {persName};
		\node [unimarc, right=of persName] (forename) {forename};
		\node [unimarc, below=of forename] (surname) {surname};
		\node [unimarc, below=of surname] (isni) {ptr};
		\node [unimarc, below=of author] (title) {title};
		\node [empty, below=of title] (empty) {};
		\node [unimarc, below=of empty] (pubPlace) {pubPlace};
		\node [attUnimarc, right=of pubPlace] (att) {@key};
		\node [unimarc, below=of pubPlace] (publisher) {publisher};
		\node [unimarc, below=of publisher] (date) {date};
		\node [attUnimarc, right=of date] (when) {@when};
		\node [attUnimarc, below=of when] (cert) {@cert};
		
		\draw [arrow] (fileDesc) -- (sourceDesc);
		\draw [arrow] (sourceDesc) -- (bibl);
		\draw [arrow] (bibl) -- (ptr);
		\draw [arrow] (bibl) |- (author);
		\draw [arrow] (bibl) |- (title);
		\draw [arrow] (bibl) |- (pubPlace);
		\draw [arrow] (bibl) |- (publisher);
		\draw [arrow] (bibl) |- (date);
		\draw [arrow] (author) -- (persName);
		\draw [arrow] (persName) -- (forename);
		\draw [arrow] (persName) |- (surname);
		\draw [arrow] (persName) |- (isni);
		\draw [arrow, dashed] (pubPlace) -- (att);
		\draw [arrow, dashed] (date) -- (when);
		\draw [arrow, dashed] (date) |- (cert);
		\draw [arrow, dashed] (ptr) -- (target);

	\end{tikzpicture}
	
\caption{Quand la relation entre le fac-similé numérique et le document physique est établie}
\end{subfigure}
\vspace{1cm}
\caption{Les sources des données du \texttt{<bibl>} du \texttt{<sourceDesc>}}
\label{fig:dataBibl}
\end{figure}
%%%%%%%%%%%%%%%%%%

%%%%%%%%%%%%%%%%%%
\begin{figure}[htp]
\centering
% legend
\begin{subfigure}[b]{\textwidth}\centering
	\begin{tikzpicture}[node distance=1cm]
		\matrix [draw] {
			\node [iiif,label=right:manifest IIIF] {};\\
			\node [unimarc,label=right:Unimarc] {};\\
			\node [sudoc,label=right:SUDOC] {}; \\
			\node [config,label=right:fichier de config] {}; \\
			\node [files,label=right:fichiers traités] {}; \\
			\node [input,label=right:saisie] {}; \\
			\node [elemDefault,label=right:par défaut] {}; \\
		};
	\end{tikzpicture}
\end{subfigure}

\vspace{1cm}

% <msDesc> IIIF
\begin{subfigure}[b]{\textwidth}\centering
	\begin{tikzpicture}[node distance=0.25cm]

		\node [elem] (fileDesc) {fileDesc};
		\node [elem, right=of fileDesc] (sourceDesc) {sourceDesc};
		\node [elem, right=of sourceDesc] (msDesc) {msDesc};
		\node [elem, right=of msDesc] (msIdentifier) {msIdentifier};
		\node [elem, right=of msIdentifier] (country) {country (\textit{vide})};
		\node [elem, below=of country] (settlement) {settlement (\textit{vide})};
		\node [iiif, below=of settlement] (repository) {repository};
		\node [iiif, below=of repository] (idno) {idno};
		\node [elem, below=of idno] (altIdentifier) {altIdentifier};
		\node [files, right=of altIdentifier] (idno2) {idno};
		\node [elem, below=of altIdentifier] (objectDesc) {objectDesc};
		\node [elem, left=of objectDesc] (physDesc) {physDesc};
		\node [elem, right=of objectDesc] (p) {p (\textit{vide})};
		
		\draw [arrow] (fileDesc) -- (sourceDesc);
		\draw [arrow] (sourceDesc) -- (msDesc);
		\draw [arrow] (msDesc) -- (msIdentifier);
		\draw [arrow] (msIdentifier) -- (country);
		\draw [arrow] (msIdentifier) |- (settlement);
		\draw [arrow] (msIdentifier) |- (repository);
		\draw [arrow] (msIdentifier) |- (idno);
		\draw [arrow] (msIdentifier) |- (altIdentifier);
		\draw [arrow] (altIdentifier) -- (idno2);
		\draw [arrow] (msDesc) |- (physDesc);
		\draw [arrow] (physDesc) -- (objectDesc);
		\draw [arrow] (objectDesc) -- (p);

	\end{tikzpicture}
	
\caption{Quand la relation entre le fac-similé numérique (renseigné par le \textit{manifest} \acrshort{IIIF} et le document physique (renseigné par catalogue général de la \acrshort{BNF}) n'est pas établie}
\end{subfigure}

\vspace{1cm}

% <msDesc> SRU
\begin{subfigure}[b]{\textwidth}\centering
	\begin{tikzpicture}[node distance=0.25cm]

		\node [elem] (fileDesc) {fileDesc};
		\node [elem, right=of fileDesc] (sourceDesc) {sourceDesc};
		\node [elem, right=of sourceDesc] (msDesc) {msDesc};
		\node [elem, right=of msDesc] (msIdentifier) {msIdentifier};
		\node [elem, right=of msIdentifier] (country) {country};
		\node [attUnimarc, right=of country] (att) {@key};
		\node [sudoc, below=of country] (settlement) {settlement};
		\node [sudoc, below=of settlement] (repository) {repository};
		\node [unimarc, below=of repository] (idno) {idno};
		\node [elem, below=of idno] (altIdentifier) {altIdentifier};
		\node [files, right=of altIdentifier] (idno2) {idno};
		\node [elem, below=of altIdentifier] (objectDesc) {objectDesc};
		\node [elem, left=of objectDesc] (physDesc) {physDesc};
		\node [unimarc, right=of objectDesc] (p) {p};
		
		\draw [arrow] (fileDesc) -- (sourceDesc);
		\draw [arrow] (sourceDesc) -- (msDesc);
		\draw [arrow] (msDesc) -- (msIdentifier);
		\draw [arrow] (msIdentifier) -- (country);
		\draw [arrow, dashed] (country) -- (att);
		\draw [arrow] (msIdentifier) |- (settlement);
		\draw [arrow] (msIdentifier) |- (repository);
		\draw [arrow] (msIdentifier) |- (idno);
		\draw [arrow] (msIdentifier) |- (altIdentifier);
		\draw [arrow] (altIdentifier) -- (idno2);
		\draw [arrow] (msDesc) |- (physDesc);
		\draw [arrow] (physDesc) -- (objectDesc);
		\draw [arrow] (objectDesc) -- (p);

	\end{tikzpicture}
	
\caption{Quand la relation entre le fac-similé numérique et le document physique est établie}
\end{subfigure}

\vspace{1cm}

\caption{Les sources des données du \texttt{<msDesc>} du \texttt{<sourceDesc>}}
\label{fig:dataMsDesc}
\end{figure}
%%%%%%%%%%%%%%%%%%
\begin{figure}[htp]\centering
\begin{tabularx}{\textwidth}
{	| >{\arraybackslash}X 
	| >{\arraybackslash}X 
	| >{\arraybackslash}X |}
\hline
élément ou attribut \acrshort{TEI} & \textit{manifest} \acrshort{IIIF} & données \Gls{unimarc}\\
\hline \hline
\texttt{sourceDesc/} \texttt{msDesc/} \texttt{msIdentifier[@key]} & -- & \texttt{datafield[@tag="801"]/} \texttt{subfield[@code="a"]} \\ \hline
\texttt{sourceDesc/} \texttt{msDesc/} \texttt{msIdentifie} \texttt{repository} & \texttt{"Repository"} & donnée du site \acrshort{SUDOC} en utilisant le numéro RCR du \texttt{datafield[@tag="930"]/} \texttt{subfield[@code="b"]} \\ \hline
\texttt{sourceDesc/} \texttt{msDesc/} \texttt{msIdentifier} \texttt{idno} & \texttt{"Shelfmark"} & \texttt{datafield[@tag="930"]/} \texttt{subfield[@code="a"]} \\ \hline
\texttt{sourceDesc/} \texttt{msDesc/} \texttt{physDesc/} \texttt{objectDesc/} \texttt{p} & -- & \texttt{datafield[@tag="200"]/} \texttt{subfield[@code="b"]} \\ \hline

\end{tabularx}
\caption{\textit{Mapping} des données du \texttt{<msDesc>} du \texttt{<sourceDesc>}}
\label{tab:sourceDescMS}
\end{figure}
%%%%%%%%%%%%%%%%%%


\subsection{Les sources des données du \texttt{<profileDesc>}}
Selon notre modélisation, qui est justifiée dans le chapitre~\ref{chap:metadata}, le \texttt{<profileDesc>} est très minimal. Il ne présente que de la donnée sur la langue utilisée dans le document source. Dans sa version actuelle, l'application \texttt{alto2tei} ne supporte qu'une seule langue. Néanmoins, malgré la simplicité conceptuelle d'un tel \texttt{<profileDesc>}, son élément \texttt{<language>} compte sur deux sources de données. Le mot balisé dans l'élément \texttt{<language>} vient du \textit{manifest} \acrshort{IIIF} puisque la langue est l'une des données déterminée essentielle et recherchée lors du traitement du \textit{manifest}. La valeur de l'attribut \texttt{@ident}, par contre, est récupérée depuis le catalogue général de la \acrshort{BNF} parce qu'elle est une clef standardisée pour la langue utilisée. La Figure~\ref{fig:dataPS} visualise les deux sources de données pour le \texttt{<profileDesc>}. Le Tableau~\ref{tab:profileDesc} précise la localisation de ces deux données.

\begin{figure}[htp]
\centering
% legend
\begin{subfigure}[b]{0.3\textwidth}\centering
\begin{tikzpicture}
\matrix [draw] {
	\node [iiif,label=right:manifest IIIF] {};\\
	\node [unimarc,label=right:Unimarc] {};\\
	\node [sudoc,label=right:SUDOC] {}; \\
	\node [config,label=right:fichier de config] {}; \\
	\node [files,label=right:fichiers traités] {}; \\
	\node [input,label=right:saisie] {}; \\
	\node [elemDefault,label=right:par défaut] {}; \\
};
\end{tikzpicture}
\end{subfigure}

\vspace{1cm}

% <profileDesc> IIIF
\begin{subfigure}[b]{\textwidth}\centering
\begin{tikzpicture}

\node [elem] (fileDesc) {fileDesc};
\node [elem, right=of fileDesc] (profileDesc) {profileDesc};
\node [elem, right=of profileDesc] (langUsage) {langUsage};
\node [iiif, right=of langUsage] (language) {language};
\node [attUnimarc, right=of language] (att) {@ident};

\draw [arrow] (fileDesc) -- (profileDesc);
\draw [arrow] (profileDesc) -- (langUsage);
\draw [arrow] (langUsage) -- (language);
\draw [arrow, dashed] (language) -- (att);

\end{tikzpicture}
\end{subfigure}

\caption{Les sources des données du \texttt{<profileDesc>}}
\label{fig:dataPS}
\end{figure}


\begin{figure}[htp]\centering
\begin{tabularx}{\textwidth}
{	| >{\arraybackslash}X 
	| >{\arraybackslash}X 
	| >{\arraybackslash}X |}
\hline
élément ou attribut \acrshort{TEI} & \textit{manifest} \acrshort{IIIF} & données \Gls{unimarc}\\
\hline \hline
\texttt{profileDesc/} \texttt{langUsage/} \texttt{language} & \texttt{"Language"} & -- \\ \hline
\texttt{profileDesc/} \texttt{langUsage/} \texttt{language[@ident]} & -- & \texttt{datafield[@tag="101"/} \texttt{subfield[@code="a"]} \\ \hline
\end{tabularx}
\caption{\textit{Mapping} des données du \texttt{<profileDesc>}}
\label{tab:profileDesc}
\end{figure}

\subsection{Les sources des données du \texttt{<encodingDesc>}}
La description de l'encodage et de l'appareil qui a fait la transcription sont essentiels. Sans eux, la transcription présentée dans la ressource ne pourra pas être reproduite. Et sans pouvoir reproduire la transcription, les scientifiques ne peuvent pas la vérifier. Pouvoir reproduire des résultats scientifiques est fondamental à la recherche.

Le \texttt{<encodingDesc>} contient deux descendants directs. Dans un premier temps, il contient de l'information sur l'engin qui a mis en œuvre les modèles \acrshort{HTR} (\texttt{<application>}). Le pipeline \textit{Gallic(orpor)a} attend toujours l'engin \textit{Kraken} et la version actuelle de l'application \texttt{alto2tei} ne renseigne que sur les versions de cet engin~; la version est donnée dans l'élément \texttt{<appInfo>}. Quand l'application \texttt{alto2tei} est démarré depuis la ligne de commande, un paramètre obligatoire déclare quelle version de \textit{Kraken} a été utilisée pour créer les fichiers \acrshort{XML} \acrshort{ALTO} qu'elle va traiter. L'application \texttt{alto2tei} tient ce paramètre et l'utilise pour renseigner sur la version de l'application \textit{Kraken} dans l'élément \texttt{<application>}. Si l'application est intégrée dans le pipeline \textit{Gallic(orpor)a}, la version de \textit{Kraken} est récupérée depuis la liste de librairies installées pour appliquer les modèles \acrshort{HTR}.

Dans un deuxième temps, le \texttt{<encodingDesc>} s'appuie sur les fichiers \acrshort{XML} \acrshort{ALTO} que l'application \texttt{alto2tei} traite. L'application analyse les étiquettes de ligne et de zone que les modèles \acrshort{HTR} ont appliquées. Ensuite, elle compare les étiquettes trouvées avec une liste des étiquettes du vocabulaire \textit{SegmOnto}. Pour chaque type d'étiquette utilisé dans les fichiers \acrshort{XML} \acrshort{ALTO} qui appartient au vocabulaire \textit{SegmOnto}, l'application \texttt{alto2tei} la déclare dans la bonne catégorie de l'élément \texttt{<taxonomy>} du \texttt{<classDecl>}, voir la sous-figure~\ref{fig:classDecl}.

En outre, puisque l'équipe du projet \textit{Gallic(orpor)a} a choisi de s'appuyer sur le vocabulaire \textit{SegmOnto} (cf. Chap.~\ref{chap:segmonto}), certains éléments du \texttt{<classDecl>} sont encodés en dur. La taxonomie déclarée dans l'identifiant de l'élément \texttt{<taxonomy>} ainsi que le titre attribué à la taxonomie (\texttt{<title>}) est toujours \og{}SegmOnto\fg{}. De plus, le \textit{pointer} (\texttt{<ptr>}) qui indique l'URL où se trouvent des informations mises à jour sur le vocabulaire utilisé est toujours celui du projet \textit{SegmOnto}. Ces données ne sont pas personnalisables. En plus, bien que les descendants des classes (\texttt{<category>}) des lignes et des zones soient informés par les fichiers \acrshort{ALTO}, les données encodées dans les éléments viennent plutôt d'une liste que l'application \texttt{alto2tei} maintient. Cette liste encodée en dur porte l'URL de toute zone et toute ligne sur le site web du projet \textit{SegmOnto}.

\begin{figure}[htp]
% legend
\begin{subfigure}[b]{\textwidth}
\centering
\begin{tikzpicture}
\matrix [draw] {
	\node [iiif,label=right:manifest IIIF] {};\\
	\node [unimarc,label=right:Unimarc] {};\\
	\node [sudoc,label=right:SUDOC] {}; \\
	\node [config,label=right:fichier de config] {}; \\
	\node [files,label=right:fichiers traités] {}; \\
	\node [input,label=right:saisie] {}; \\
	\node [elemDefault,label=right:par défaut] {}; \\
};
\end{tikzpicture}
\end{subfigure}

\vspace{1cm}

% <appInfo>
\begin{subfigure}[b]{\textwidth}\centering
\begin{tikzpicture}[node distance=0.5cm]

\node [elem] (fileDesc) {fileDesc};
\node [elem, right=of fileDesc] (encodingDesc) {encodingDesc};
\node [elem, right=of encodingDesc] (appInfo) {appInfo};
\node [elem, right=of appInfo] (application) {application};
\node [attDefault, right=of application] (ident) {@ident};
\node [attInput, below=of ident] (version) {@version};
\node [empty, below=of version] (empty) {};
\node [elemDefault, right=of empty] (label) {label};
\node [elemDefault, below=of label] (ptr) {ptr};

\draw [arrow] (fileDesc) -- (encodingDesc);
\draw [arrow] (encodingDesc) -- (appInfo);
\draw [arrow] (appInfo) -- (application);
\draw [arrow, shorten <=0.5cm] (application) |- (label);
\draw [arrow, shorten <=0.5cm] (application) |- (ptr);
\draw [arrow, dashed] (application) -- (ident);
\draw [arrow, dashed] (application) |- (version);

\end{tikzpicture}
\caption{La source de données du \texttt{<appInfo>}}
\end{subfigure}

\vspace{1cm}

% <classDecl>
\begin{subfigure}[b]{\textwidth}\centering
\begin{tikzpicture}[node distance=0.5]

\node [elem] (fileDesc) {fileDesc};
\node [elem, right=of fileDesc] (encodingDesc) {encodingDesc};
\node [elem, right=of encodingDesc] (classDecl) {classDecl};
\node [elem, right=of classDecl] (taxonomy) {taxonomy};
\node [attDefault, right=of taxonomy] (taxid) {@xml:id};
\node [elemDefault, below=of taxid] (titleBibl) {title};
\node [elem, left=of titleBibl] (bibl) {bibl};
\node [elemDefault, below=of titleBibl] (ptrBibl) {ptr};
\node [empty, below=of bibl] (emp) {};
\node [elem, below=of emp] (category) {category};
\node [files, right=of category] (catDesc) {catDesc};
\node [files, right=of catDesc] (title) {title};
\node [files, below=of title] (ptr) {ptr};

\draw [arrow] (fileDesc) -- (encodingDesc);
\draw [arrow] (encodingDesc) -- (classDecl);
\draw [arrow] (classDecl) |- (bibl);
\draw [arrow] (classDecl) -- (taxonomy);
\draw [arrow, dashed] (taxonomy) -- (taxid);
\draw [arrow] (bibl) -- (titleBibl);
\draw [arrow] (ptrBibl) -| (bibl);
\draw [arrow] (category) -|  (classDecl);
\draw [arrow] (ptr) -| (catDesc);
\draw [arrow] (category) -- (catDesc);
\draw [arrow] (catDesc) -- (title);

\end{tikzpicture}
\caption{La source de données du \texttt{<classDecl>}}
\label{fig:classDecl}
\end{subfigure}

\caption{Les sources de données du \texttt{<encodingDesc>}}
\label{fig:dataED}
\end{figure}


\end{document}
\documentclass[../main.tex]{subfiles}